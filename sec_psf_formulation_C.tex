

A second possible approach lies in measuring the structure functions
$D_{\rm \phi_{inst,sci}}(\vec{s}, \vec{r} +\vec{s} )$ 
and
$D_{\rm \phi_{ao}}(\vec{s}, \vec{r} + \vec{s})$
and using the measured values directly in Equation
\ref{eqn:lgsotf}.  The static, field-dependent instrumental wavefront
error, $\phi_{\rm inst,sci}(\vec{r})$, may be measured using a technique 
such as phase diversity
\begin{notes}
[CITE our work]
\end{notes}
.  The
structure function may then be formed by brute force numerical
calculation via
\begin{equation}
\label{eqn:strfn_inst}
D_{\rm \phi_{inst,sci}}(\vec{r}_{1}, \vec{r}_{2})  = 
\left\langle \left\{\phi_{\rm inst,sci}\left(\vec{r}_{1}\right) - 
\phi_{\rm inst,sci}\left(\vec{r}_{2}\right)\right\}^{2}\right\rangle
\end{equation}
Similarly, the statistical properties of $\phi_{\rm
  ao}\left(\vec{r}\right)$ may be characterized via modelling
of the adaptive optics system.  Such a characterization would involve
understanding the contribution to the structure function $D_{\rm
  \phi_{ao}}(\vec{r}_{1},\vec{r}_{2})$ from measurement errors,
servo and fitting errors for specific hardware and under specific
observing conditions.  For example, the statistical properties of 
high-order fitting error are determined by the turbulence profile and
actuator pitch on the deformable mirror
\begin{notes}
[CITE].
\end{notes}
Similarly, statistical
properties of high-order measurement error are dictated by the
geometry of the wavefront sensor, noise and diffusion properties of
the wavefront sensor detector, and by the guide star brightness.  Finally,
the statistical properties of high-order servo errors are dictated by the
turbulence and wind profiles and by the latency in the real time
controller.  If the four dimensional structure function $D_{\rm
  \phi_{ao}}(\vec{r}_{1},\vec{r}_{2})$ may be modelled
successfully, then the OTF may be computed
directly from Equation \ref{eqn:lgsotf}.  Modelling of $D_{\rm
  \phi_{ao}}(\vec{r}_{1},\vec{r}_{2})$ at Keck is beyond 
the scope of this project and is being pursued in a parallel effort funded 
through the NSF ATI program 
\begin{notes}
[PI: Wizinowich, NSF-ATI number].
\end{notes}
The results of this modelling may demonstrate that 
$D_{\rm \phi_{ao}}(\vec{r}_{1},\vec{r}_{2})$ is nearly
stationary.  This might be argued based on the fact that fitting,
servo and measurement errors tend to decorrelate at separations of a
single subaperture, and do so uniformly over the pupil plane.  This
leads to structure functions that are stationary.  Were this
assumption to hold, one could write Equation \ref{eqn:lgsotf} as
\begin{equation}\label{eqn:lgsotfb}
\begin{aligned}
{\rm OTF}^{\rm LGS}_{\rm sci}(\vec{r}) = 
& \exp{
\left\{ -\frac{1}{2} \bar{D}_{\rm \phi_{\rm ao}}(\vec{r}) \right\}}  \\
& \int 
W \left( \frac{\vec{s}}{R} \right)
W \left( \frac{\vec{r} + \vec{s}}{R} \right)
\exp{ \left\{ -\frac{1}{2} \left[
D_{\rm \phi_{apl}}(\vec{s}, \vec{r} + \vec{s}) +
D_{\rm \phi_{inst,sci}}(\vec{s}, \vec{r} + \vec{s})
\right] \right\} }
\; d\vec{s} 
\end{aligned}
\end{equation}
Naturally this stationarity assumption must be validated against
performance of real hardware, and its validity is dictated by the
modelling accuracy required for the astronomical application.  

Finally, one could postulate a model that assumes the instrumental
structure function is stationary, so that
\begin{equation}\label{eqn:lgsotfc}
\begin{aligned}
{\rm OTF}^{\rm LGS}_{\rm sci}(\vec{r}) = 
& \exp{ \left\{ -\frac{1}{2}
\left[ \bar{D}_{\rm \phi_{\rm ao}}(\vec{r}) \right] \right\} } 
\exp{ \left\{ -\frac{1}{2}
\left[ \bar{D}_{\rm \phi_{\rm inst,sci}}(\vec{r}) \right] \right\} }  \\
& \int 
W \left( \frac{\vec{s}}{R} \right)
W \left( \frac{\vec{r} + \vec{s}}{R} \right)
\exp{ \left\{ -\frac{1}{2} \left[
D_{\rm \phi_{apl}}(\vec{s},\vec{r} + \vec{s})
\right] \right\} }
\; d\vec{s} 
\end{aligned}
\end{equation}
Example PSF and OTFs are shown in Figure \ref{fig:lgs_atm_inst} for
the case of an on-axis LGS and TTS both with atmospheric
anisoplanatism and both with and without stationary instrumental 
aberrations. This approximation would be suitable when 
$\phi_{\rm inst,sci} \left( \vec{r} \right) \ll 
\phi_{\rm apl} \left( \vec{r} \right)$. Ultimately, 
this approximation is adopted for AIROPA.

