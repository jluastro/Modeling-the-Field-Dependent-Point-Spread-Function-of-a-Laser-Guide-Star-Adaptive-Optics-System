
The validity of the approximations in Equations \ref{eqn:lgsotfa},
\ref{eqn:lgsotfb} and \ref{eqn:lgsotfc} to the model in Equation \ref{eqn:lgsotf}
will depend signficantly on the character of the adaptive optics
system, the optical properties of the instrumentation, and the
importance of these terms relative to the degree of anisoplanatism
inherent in the observational scenario.  The importance of
anisoplanatism itself depends on many factors: the distribution of the
sources with respect to the laser and tip-tilt guide star, the
turbulence profile, zenith angle, observing wavelength, and aperture
diameter.  
\begin{notes}
Should we start to introduce the concept that the on-axis part
comes from the data itself? 
\end{notes}

The state of our knowledge regarding $D_{\rm
  \phi_{ao}}(\vec{s},\vec{r} + \vec{s})$ and
$D_{\rm \phi_{inst,sci}}(\vec{s},\vec{r} +
\vec{s})$ at Keck Observatory is currently evolving.  This
report presents the methodology to compute $D_{\rm
  \phi_{apl}}(\vec{s},\vec{r})$.  Knowledge of each of
these quantities is necessary to formulate the prediction of
${\rm OTF}^{\rm LGS}_{\rm sci}(\vec{r})$ in Equation \ref{eqn:lgsotf}, or to
utilize the approximations in Equations \ref{eqn:lgsotfa}, \ref{eqn:lgsotfb}
and \ref{eqn:lgsotfc}.

Proceeding along this line of inquiry, the anisoplanatic structure
function may be expressed in terms of the quantities defined in
Equation \ref{eqn:phidefs} as
\begin{equation}\label{eqn:lgsstrfnexpansion}
\begin{split}
D_{\phi_{\rm apl}}(\vec{r}_{1},\vec{r}_{2}) & =
\left\langle \left\{\phi_{\rm apl}\left(\vec{r}_{1}\right) - 
\phi_{\rm apl}\left(\vec{r}_{2}\right)\right\}^{2}\right\rangle \\
& 
=\left\langle \left\{
\left[\phi_{\rm sci}\left(\vec{r}_{1}\right) - \phi^{T}_{\rm ttgs}\left(\vec{r}_{1}\right) - 
\phi_{\rm lgs}\left(\vec{r}_{1}\right) + \phi^{T}_{\rm lgs}\left(\vec{r}_{1}\right)\right] -
\vphantom{\left\langle\left\{\right\}^{2}\right\rangle}
\right.
\right.
\\
& \quad
\left.
\left.
\left[\phi_{\rm sci}\left(\vec{r}_{2}\right) - \phi^{T}_{\rm ttgs}\left(\vec{r}_{2}\right) - 
\phi_{\rm lgs}\left(\vec{r}_{2}\right) + \phi^{T}_{\rm lgs}\left(\vec{r}_{2}\right)\right]
\right\}^{2}\right\rangle  
\end{split}
\end{equation}
This expression may be written as the sum of 36 two-point phase
covariance functions.  For example, the term $\left\langle\phi_{\rm
  sci}\left(\vec{r}_{1}\right)\phi_{\rm
  sci}\left(\vec{r}_{2}\right)\right\rangle$ represents the
piston-removed wavefront phase covariance on the science target.  
For the special
case $\vec{r}_{1}=\vec{r}_{2}\equiv\vec{r}$, this
term simplifies to $\left\langle \left\{\phi_{\rm
  sci}\left(\vec{r}\right)\right\}^{2}\right\rangle$,
representing the target piston-removed wavefront phase variance as a
function of the pupil plane coordinate $\vec{r}$.  As a
concrete example of the utility of this function, we could measure the
piston-removed wavefront phase using a Shack Hartmann sensor and
compute the variance at each subaperture in the pupil plane.  The
function $\left\langle\left\{\phi_{\rm
  sci}\left(\vec{r}\right)^{2}\right\}\right\rangle$ would represent
the Komolgorov prediction corresponding to our measurement.

\section{Two-Point Covariance Functions}
\label{sec:covariance}

The key elements required for the prediction of the structure function
in Equation \ref{eqn:lgsstrfnexpansion} and hence the optical transfer
function in Equation \ref{eqn:lgsotfa} are the two-point covariance
functions that appear in Equation \ref{eqn:lgsstrfnexpansion}.  The
following three such functions are required to compute the structure
function.
\begin{align}
\left\langle \phi_{a}\left(\vec{r}_{1}\right) \phi_{b}\left(\vec{r}_{2}\right) \right\rangle &\quad\quad\quad {\rm Piston\text{-}removed\;phase\;covariance} \label{eqn:prphase} \\
\left\langle \phi^{T}_{a}\left(\vec{r}_{1}\right) \phi^{T}_{b}\left(\vec{r}_{2}\right) \right\rangle &\quad\quad\quad {\rm Tilt\;phase\;covariance} \label{eqn:tphase} \\
\left\langle \phi_{a}\left(\vec{r}_{1}\right) \phi^{T}_{b}\left(\vec{r}_{2}\right) \right\rangle &\quad\quad\quad {\rm Piston\text{-}removed\;/\;tilt\;cross\;covariance} \label{eqn:tprphase} 
\end{align}
In this expression the subscripts $a$ and $b$ label the two sources,
while the vectors $\vec{r}_{1}$ and $\vec{r}_{2}$
represent coordinates in the telescope aperture.  
The derivation of these quanitites assuming a Komolgorov
turbulence spectrum appears in
\cite{Tyler:1994a} and a complementary approach appears in 
\cite{Sasiela:2012}.
For completeness, we outline the derivation and 
link to specific code implementations within the AIROPA
pacakge in Appendix \ref{app:kolmogorov_deriv}. 


Equation \ref{eqn:prphase} represents the two point piston removed
phase covariance function for two sources at arbitrary ranges and
locations within the field.  The parameters that enter into the
calculation of this quantity are the wavenumber $k$, aperture diameter
$D$, turbulence profile $C_{n}^{2}(z)$, source ranges
$\mathcal{R}_{a}$ and $\mathcal{R}_{b}$, and their locations
$\vec{r}_{a}$ and $\vec{r}_{b}$ in the field.  All of
these parameters are known in a specific observing geometry such as
that shown in Figure \ref{geo} with the exception of the turbulence
profile $C_{n}^{2}(z)$.  
On Maunakea, a $C_{n}^{2}(z)$ profile is measured in real-time
using a MASS/DIMM facility and is made available through the Mauna Kea Weather
Center archive\footnote{\url{http://mkwc.ifa.hawaii.edu/current/seeing/}}. 
When no real-time profile is available, 
a standard $C_{n}^{2}(z)$ profile may be assumed.  Likewise,
Equations \ref{eqn:tphase} and \ref{eqn:tprphase} respectively represent
the tilt covariance and piston-removed / tilt phase cross covariance
for two sources at arbitrary ranges and and locations within the
field.  These quantities may be computed using the same inputs.

Calculation of these covariance functions requires two numerical
integrations over the unit circle (i.e over $\vec{\rho}_{1}$
and $\vec{\rho}_{2}$).  Assuming this integration is performed
using $\mathcal{N}$ samples across the pupil, then computation of
$\left\langle \vphantom{\hat{\phi}_{a}}
\phi_{a}\left(R\vec{\rho}_{1}\right)
\phi_{b}\left(R\vec{\rho}_{2}\right)\right\rangle$ is an
$\mathcal{O}(\mathcal{N}^{4})$ calculation.  Recall from Equation
\ref{lgsotf} that calculation of the four dimensional structure
function $D_{\rm apl}\left(\vec{r}_{1},
\vec{r}_{2}\right)$ must be performed for
$\mathcal{O}(\mathcal{N}^{4})$ values of $\vec{r}_{1}$ and
$\vec{r}_{2}$, each of which requires computation of 36
covariance functions.  This indicates that computation of the optical
transfer function is an $\mathcal{O}(36\mathcal{N}^{8})$ process.
Such a calculation is not a realistic proposition for general use in
astronomical data reduction.

\section{Computational Formulation}
\label{sec:compcovariance}


Arroyo \cite{}

Tyler addresses the issue of computability by providing simplifying
expressions for each of the integrals that appear in Equations
\ref{eqn:prphase}, \ref{eqn:tphase} and \ref{eqn:tprphase} .  This report
adheres to Tyler's Geugenbauer polynomial formalism, which is
presented in Appendix A3\cite{1994JOSAA..11..409T}.  The piston
removed phase covariance is given by
\begin{equation}\label{phasecovariance}
\left\langle \phi_{a}\left(R\vec{\rho}_{1}\right)  \phi_{b}\left(R\vec{\rho}_{2}\right)\right\rangle = 
\Xi D^{5/3} k^{2} \left\{
\mathcal{A}_{ab}\left(\vec{\rho}_{1}\right) + 
\mathcal{B}_{ab}\left(\vec{\rho}_{2}\right) - 
\mathcal{C}_{ab}\left(\vec{\rho}_{1},\vec{\rho}_{2}\right) - 
\mathcal{D}_{ab}\right\}
\end{equation}
and the tilt covariance is given by
\begin{equation}\label{tiltcovariance}
\begin{split}
\left\langle \phi^{T}_{a}\left(R\vec{\rho}_{1}\right)  \phi^{T}_{b}\left(R\vec{\rho}_{2}\right)\right\rangle & = 
\Xi D^{5/3} k^{2} \left\{
\left[\vec{\rho}_{1} \cdot \vec{\rho}_{2} \right] \mathcal{E}_{ab} + 
\mathcal{F}_{ab}\left(\vec{\rho}_{1}, \vec{\rho}_{2}\right)
\right\} 
\end{split}
\end{equation}
Finally, the cross covariance is given by
\begin{equation}\label{crosscovariance}
\begin{split}
\left\langle \phi_{a}\left(R\vec{\rho}_{1}\right)  \phi^{T}_{b}\left(R\vec{\rho}_{2}\right)\right\rangle & = 
\Xi D^{5/3} k^{2} \left\{
\left[\vec{\rho}_{1} \cdot \vec{\rho}_{2} \right] \mathcal{G}_{ab}\left(\vec{\rho}_{1}\right) + 
\mathcal{H}_{ab}\left(\vec{\rho}_{1}, \vec{\rho}_{2}\right) + 
\mathcal{I}_{ab}\left(\vec{\rho}_{2}\right)
\right\} 
\end{split}
\end{equation}
In these expressions, the following definitions have been employed.
\begin{equation}\label{caliph_A}
\mathcal{A}_{ab}\left(\vec{\rho}_{1}\right) = 
\int_{0}^{\mathcal{R}_{\rm min}} dz \; C_{n}^{2}\left(z\right) \left( 1 - \frac{z}{\mathcal{R}_{a}}\right)^{5/3}\; 
\left[Q_{ab}\left(z\right)\right]^{5/3} 
F_{1}\left(\frac{\left\vert\vec{\rho}_{1} + \vec{\Omega}_{ab}\left(z\right)\right\vert}{Q_{ab}\left(z\right)}\right)
\end{equation}
\begin{equation}\label{caliph_B}
\mathcal{B}_{ab}\left(\vec{\rho}_{2}\right) = 
\int_{0}^{\mathcal{R}_{\rm min}} dz \; C_{n}^{2}\left(z\right) \left( 1 - \frac{z}{\mathcal{R}_{a}}\right)^{5/3}\; 
F_{1}\left(\left\vert Q_{ab}\left(z\right)\vec{\rho}_{2} - \vec{\Omega}_{ab}\left(z\right)\right\vert\right)
\end{equation}
\begin{equation}\label{caliph_C}
\mathcal{C}_{ab}\left(\vec{\rho}_{1}, \vec{\rho}_{2}\right) = 
\int_{0}^{\mathcal{R}_{\rm min}} dz \; C_{n}^{2}\left(z\right) \left( 1 - \frac{z}{\mathcal{R}_{a}}\right)^{5/3}\; 
\left\vert\vec{\rho}_{1} - Q_{ab}\left(z\right)\vec{\rho}_{2} + \vec{\Omega}_{ab}\left(z\right)\right\vert^{5/3}
\end{equation}
\begin{equation}\label{caliph_D}
\mathcal{D}_{ab} = 
\int_{0}^{\mathcal{R}_{\rm min}} dz \; C_{n}^{2}\left(z\right) \left( 1 - \frac{z}{\mathcal{R}_{a}}\right)^{5/3}\; 
F_{2}\left(Q_{ab}\left(z\right),\left\vert\vec{\Omega}_{ab}\left(z\right)\right\vert\right)
\end{equation}
\begin{equation}\label{caliph_E}
\begin{split}
\mathcal{E}_{ab} = 
16 \int_{0}^{\mathcal{R}_{\rm min}} dz & \; C_{n}^{2}\left(z\right) \left( 1 - \frac{z}{\mathcal{R}_{a}}\right)^{5/3}\; 
\left[Q_{ab}\left(z\right)\right]^{-3}
\\ 
& \left\{\int_{0}^{{\rm max}\left(Q_{ab}\left(z\right)-\left\vert \vec{\Omega}_{ab} \left(z\right)\right\vert,0\right)} d\rho \; \rho^{3} \; \hat{G}\left(\rho\right) + 
\right.
\\ 
& 
\quad\quad
\frac{1}{\pi} \int^{Q_{ab}\left(z\right)+\left\vert \vec{\Omega}_{ab}\left(z\right)\right\vert}_{\left\vert Q_{ab}\left(z\right)-\left\vert \vec{\Omega}_{ab}\left(z\right)\right\vert\right\vert}
d\rho \; \rho^{3} \; \hat{G}\left(\rho\right) L\left(\left\vert\vec{\Omega}_{ab}\left(z\right)\right\vert, \rho, Q_{ab}\left(z\right)\right) \\
& 
\quad\quad\quad\quad
\left.
\sqrt{1 - \left[L\left(\left\vert\vec{\Omega}_{ab}\left(z\right)\right\vert, \rho, Q_{ab}\left(z\right)\right)\right]^{2}}
\vphantom{\int_{0}^{{\rm max}\left(Q_{ab}\left(z\right)-\left\vert \vec{\Omega}_{ab} \left(z\right)\right\vert,0\right)}}
\right\} 
\end{split}
\end{equation}
\begin{equation}\label{caliph_F}
\begin{split}
\mathcal{F}_{ab}\left(\vec{\rho}_{1}, \vec{\rho}_{2}\right) = 
\frac{32}{\pi}& \int_{0}^{\mathcal{R}_{\rm min}} dz \; C_{n}^{2}\left(z\right) \left( 1 - \frac{z}{\mathcal{R}_{a}}\right)^{5/3}
\left[\vec{\rho}_{1} \cdot \vec{\Omega}_{ab}\left(z\right) \right]
\left[\vec{\rho}_{2} \cdot \vec{\Omega}_{ab}\left(z\right) \right] \\
&
\int^{Q_{ab}\left(z\right)+\left\vert \vec{\Omega}_{ab}\left(z\right)\right\vert}_{\left\vert Q_{ab}\left(z\right)-\left\vert \vec{\Omega}_{ab}\left(z\right)\right\vert\right\vert}
d\rho \; \rho^{2} \; \hat{G}\left(\rho\right) 
\left[L\left(\left\vert\vec{\Omega}_{ab}\left(z\right)\right\vert, \rho, Q_{ab}\left(z\right)\right) - \frac{\left\vert \vec{\Omega}_{ab}\left(z\right) \right\vert}{\rho}\right]\\
& 
\quad\quad\quad\quad\quad\quad
\left[1 - \left[L\left(\left\vert\vec{\Omega}_{ab}\left(z\right)\right\vert, \rho, Q_{ab}\left(z\right)\right)\right]^{2}\right]^{1/2}
\end{split}
\end{equation}
\begin{equation}\label{caliph_G}
\mathcal{G}_{ab}\left(\vec{\rho}_{1}\right) = 
4 \int_{0}^{\mathcal{R}_{\rm min}} dz \; C_{n}^{2}\left(z\right) \left( 1 - \frac{z}{\mathcal{R}_{a}}\right)^{5/3}\; 
\left[Q_{ab}\left(z\right)\right]^{2/3} 
\hat{G}\left(\left\vert \frac{\vec{\rho}_{1} + \vec{\Omega}_{ab}\left(z\right)}{Q_{ab}\left(z\right)}\right\vert\right)
\end{equation}
\begin{equation}\label{caliph_H}
\mathcal{H}_{ab}\left(\vec{\rho}_{1}, \vec{\rho}_{2}\right) = 
4 \int_{0}^{\mathcal{R}_{\rm min}} dz \; C_{n}^{2}\left(z\right) \left( 1 - \frac{z}{\mathcal{R}_{a}}\right)^{5/3}\; 
\left[Q_{ab}\left(z\right)\right]^{2/3} \left[\vec{\rho}_{1}\cdot\vec{\Omega}_{ab}\left(z\right)\right]
\hat{G}\left(\left\vert \frac{\vec{\rho}_{2} + \vec{\Omega}_{ab}\left(z\right)}{Q_{ab}\left(z\right)}\right\vert\right)
\end{equation}
\begin{equation}\label{caliph_I}
\begin{split}
\mathcal{I}_{ab}\left(\vec{\rho}_{2}\right) = & 
4 \int_{0}^{\mathcal{R}_{\rm min}} dz \; C_{n}^{2}\left(z\right) \left( 1 - \frac{z}{\mathcal{R}_{a}}\right)^{5/3}\; 
\frac{\left[\vec{\rho}_{2}\cdot\vec{\Omega}_{ab}\left(z\right)\right] }{\pi \left[Q_{ab}\left(z\right)\right]^{1/3} 
\left\vert\vec{\Omega}_{ab}\left(z\right)\right\vert} \\
& 
\quad\quad\quad\quad\quad\quad
\int^{Q_{ab}\left(z\right) +\left\vert \vec{\Omega}_{ab}\left(z\right)\right\vert}_{\left\vert Q_{ab}\left(z\right) - \left\vert\vec{\Omega}_{ab}\left(z\right)\right\vert\right\vert}
d\rho \rho^{2} \left\{1 - \left[L\left(\left\vert\vec{\Omega}_{ab}\left(z\right)\right\vert, \rho, Q_{ab}\left(z\right)\right)\right]^{2}\right\}^{1/2} 
\hat{G}\left(\rho\right)
\end{split}
\end{equation}

The following definitions have been employed in Equations \ref{caliph_A} - \ref{caliph_H}.
\begin{equation}\label{F_1_eqn}
F_{1}\left(x\right) = 
\left\{
\begin{aligned}
\frac{6}{11}\; \vphantom{F}_{2}F_{1}\left(-\frac{11}{6}, -\frac{5}{6};1; x^{2}\right) & \quad x\le 1 \\
x^{5/3}\; \vphantom{F}_{2}F_{1}\left(-\frac{5}{6}, -\frac{5}{6};2; x^{-2}\right)        & \quad x>1
\end{aligned}
\right.
\end{equation}
\begin{equation}\label{F_2_eqn}
F_{2}\left(\alpha, x\right) = \frac{2}{\pi \alpha^{2}} \int_{0}^{1+\alpha} d\rho K_{1}\left(\rho, \alpha\right)H\left(\rho, x\right)
\end{equation}
\begin{equation}\label{K_1_eqn}
K_{1}\left(\rho,\alpha\right) = 
\left\{
\begin{aligned}
&
\pi \alpha^{2} & \left\vert\rho\right\vert \le 1-\alpha \\
&
\begin{split}
\cos^{-1}&\left(t_{0}\right) - t_{0}\sqrt{1-t_{0}} + \\
&\alpha^{2}\left[\cos^{-1}\left(s_{0}/\alpha\right) - s_{0}/\alpha\sqrt{1-s_{0}/\alpha^{2}}\right] 
\end{split}
& \quad 1 - \alpha \le \left\vert\rho\right\vert  \le 1+\alpha \\
\\
& 0 & {\rm Otherwise}
\end{aligned}
\right.
\end{equation}
\begin{equation}\label{s0_t0_eqn}
\begin{aligned}
s_{0} = \frac{\rho}{2} - \frac{1 - \alpha^{2}}{2\rho} \quad & \quad t_{0} = \frac{\rho}{2} + \frac{1 - \alpha^{2}}{2\rho} 
\end{aligned}
\end{equation}
\begin{equation}\label{H_eqn}
H\left(\rho,x\right) = 
\left\{
\begin{aligned}
\rho^{8/3} \vphantom{F}_{2}F_{1}\left(-\frac{5}{6}, -\frac{5}{6};1; \left(x/\rho\right)^{-2}\right)        & \quad x\le \rho \\
\rho x^{5/3} \vphantom{F}_{2}F_{1}\left(-\frac{5}{6}, -\frac{5}{6};1; \left(\rho/x\right)^{-2}\right)        & \quad x > \rho
\end{aligned}
\right.
\end{equation}
\begin{equation}\label{G_hat_eqn}
\hat{G}\left(\alpha\right) = 
\left\{
\begin{aligned}
-\frac{5}{11} \vphantom{F}_{2}F_{1}\left(-\frac{11}{6}, \frac{1}{6};2; \alpha^{2}\right)              & \quad \alpha\le 1 \\
-\frac{5}{12} \alpha^{-1/3} \vphantom{F}_{2}F_{1}\left(-\frac{5}{6}, \frac{1}{6};3; \alpha^{-2}\right)  & \quad \alpha > 1
\end{aligned}
\right.
\end{equation}
\begin{equation}
L\left(a, b, c\right) = \frac{a^{2} + b^{2} - c^{2}}{2 a b}
\end{equation}

The above expressions appear complex, but provide significant
computational advantage over Equation \ref{eqn:prphase}.  The
calculation of Equations \ref{caliph_A} - \ref{caliph_I} require at
most the calculation of a one dimensional integral over
$C_{n}^{2}(z)$, a one dimensional integral over another variable, and
computation of a hypergeometric function.  The latter computation is
easily terminated when contributions of successive hypergeometric
terms fall below a fixed precision.  This is vastly more
computationally efficient than the $\mathcal{O}(\mathcal{N}^{4})$
calculation required to compute Equations \ref{eqn:prphase} -
\ref{eqn:tprphase} directly.

The form of Equations \ref{phasecovariance} - \ref{crosscovariance}
permit further computational advantages.  Six of the nine expressions
display a reduced dependence on $\vec{\rho}_{1}$ and/or
$\vec{\rho}_{2}$, and may be precomputed and stored for use in
computing the covariance functions over the pupil plane.

In laser guide star adaptive optics systems, one is interested in
computing the field dependent target optical transfer function
${\rm OTF}_{\rm sci}\left(\vec{r}\right)$ assuming fixed locations of
the tip tilt and laser guide stars and for a specific turbulence
profile.  For example, if one was interested in modelling a crowded
field imaged by a science camera, one would wish to compute ${\rm OTF}_{\rm
  sci}\left(\vec{r}\right)$ at many locations within the field
of view.  In such an application, one would precompute the piston
removed phase and tilt covariance functions involving the laser and
tip tilt star that appear in Equation \ref{lgsstrfnexpansion}.  One
would then iterate over locations in the field, computing only the
terms in in Equation \ref{lgsstrfnexpansion} that depend on target
location.


\section{Anisoplanatic Structure Function}
\label{sec:strfn}

%\phi_{\rm sci}\left(\vec{r}\right)
%\phi^{T}_{\rm sci}\left(\vec{r}\right) 
%\phi^{T}_{\rm ttgs}\left(\vec{r}\right)
%\phi_{\rm lgs}\left(\vec{r}\right)
%\phi^{T}_{\rm lgs}\left(\vec{r}\right)

For computational purposes it is useful to explicitly expand the
structure function in Equation \ref{lgsstrfnexpansion} and group like
terms, yielding
\begin{equation}\label{strfn_x}
\small
\begin{aligned}
D_{\phi_{\rm apl}}&(\vec{r}_{1}, \vec{r}_{2}) = 
\\ &
\left\langle\left\{\phi_{\rm sci}\left(\vec{r}_{1}\right)\right\}^{2}\right\rangle +
\left\langle\left\{\phi_{\rm sci}\left(\vec{r}_{2}\right)\right\}^{2}\right\rangle -
2\left\langle\phi_{\rm sci}\left(\vec{r}_{1}\right) \phi_{\rm sci}\left(\vec{r}_{2}\right)\right\rangle + 
\\ & 
\left\langle\left\{\phi^{T}_{\rm ttgs}\left(\vec{r}_{1}\right)\right\}^{2}\right\rangle  + 
\left\langle\left\{\phi^{T}_{\rm ttgs}\left(\vec{r}_{2}\right) \right\}^{2}\right\rangle -
2\left\langle\phi^{T}_{\rm ttgs}\left(\vec{r}_{1}\right) \phi^{T}_{\rm ttgs}\left(\vec{r}_{2}\right)\right\rangle +
\\ & 
\left\langle\left\{ \phi_{\rm lgs}\left(\vec{r}_{1}\right)\right\}^{2}\right\rangle +
\left\langle\left\{ \phi_{\rm lgs}\left(\vec{r}_{2}\right)\right\}^{2}\right\rangle -
2\left\langle\phi_{\rm lgs}\left(\vec{r}_{1}\right) \phi_{\rm lgs}\left(\vec{r}_{2}\right)\right\rangle +
\\ & 
\left\langle\left\{ \phi^{T}_{\rm lgs}\left(\vec{r}_{1}\right)\right\}^{2}\right\rangle  +
\left\langle\left\{ \phi^{T}_{\rm lgs}\left(\vec{r}_{2}\right)\right\}^{2}\right\rangle - 
2\left\langle\phi^{T}_{\rm lgs}\left(\vec{r}_{1}\right) \phi^{T}_{\rm lgs}\left(\vec{r}_{2}\right)\right\rangle +
\\ & 
2\left[-\left\langle\phi_{\rm sci}\left(\vec{r}_{1}\right) \phi^{T}_{\rm ttgs}\left(\vec{r}_{1}\right)\right\rangle + 
\left\langle\phi_{\rm sci}\left(\vec{r}_{2}\right) \phi^{T}_{\rm ttgs}\left(\vec{r}_{2}\right)\right\rangle +
\left\langle\phi_{\rm sci}\left(\vec{r}_{1}\right) \phi^{T}_{\rm ttgs}\left(\vec{r}_{2}\right)\right\rangle +
\left\langle\phi_{\rm sci}\left(\vec{r}_{2}\right) \phi^{T}_{\rm ttgs}\left(\vec{r}_{1}\right)\right\rangle\right] +
\\ & 
2\left[-\left\langle\phi_{\rm sci}\left(\vec{r}_{1}\right) \phi_{\rm lgs}\left(\vec{r}_{1}\right)\right\rangle + 
\left\langle\phi_{\rm sci}\left(\vec{r}_{2}\right) \phi_{\rm lgs}\left(\vec{r}_{2}\right)\right\rangle +
\left\langle\phi_{\rm sci}\left(\vec{r}_{1}\right) \phi_{\rm lgs}\left(\vec{r}_{2}\right)\right\rangle +
\left\langle\phi_{\rm sci}\left(\vec{r}_{2}\right) \phi_{\rm lgs}\left(\vec{r}_{1}\right)\right\rangle\right] +
\\ & 
2\left[\left\langle\phi_{\rm sci}\left(\vec{r}_{1}\right) \phi^{T}_{\rm lgs}\left(\vec{r}_{1}\right)\right\rangle -
\left\langle\phi_{\rm sci}\left(\vec{r}_{2}\right) \phi^{T}_{\rm lgs}\left(\vec{r}_{2}\right)\right\rangle -
\left\langle\phi_{\rm sci}\left(\vec{r}_{1}\right) \phi^{T}_{\rm lgs}\left(\vec{r}_{2}\right)\right\rangle -
\left\langle\phi_{\rm sci}\left(\vec{r}_{2}\right) \phi^{T}_{\rm lgs}\left(\vec{r}_{1}\right)\right\rangle\right] + 
\\ & 
2\left[\left\langle\phi^{T}_{\rm ttgs}\left(\vec{r}_{1}\right) \phi_{\rm lgs}\left(\vec{r}_{1}\right)\right\rangle - 
\left\langle\phi^{T}_{\rm ttgs}\left(\vec{r}_{2}\right) \phi_{\rm lgs}\left(\vec{r}_{2}\right)\right\rangle -
\left\langle\phi^{T}_{\rm ttgs}\left(\vec{r}_{1}\right) \phi_{\rm lgs}\left(\vec{r}_{2}\right)\right\rangle -
\left\langle\phi^{T}_{\rm ttgs}\left(\vec{r}_{2}\right) \phi_{\rm lgs}\left(\vec{r}_{1}\right)\right\rangle\right] + 
\\ & 
2\left[-\left\langle\phi^{T}_{\rm ttgs}\left(\vec{r}_{1}\right) \phi^{T}_{\rm lgs}\left(\vec{r}_{1}\right)\right\rangle + 
\left\langle\phi^{T}_{\rm ttgs}\left(\vec{r}_{2}\right) \phi^{T}_{\rm lgs}\left(\vec{r}_{2}\right)\right\rangle +
\left\langle\phi^{T}_{\rm ttgs}\left(\vec{r}_{1}\right) \phi^{T}_{\rm lgs}\left(\vec{r}_{2}\right)\right\rangle +
\left\langle\phi^{T}_{\rm ttgs}\left(\vec{r}_{2}\right) \phi^{T}_{\rm lgs}\left(\vec{r}_{1}\right)\right\rangle\right] +
\\ & 
2\left[-\left\langle\phi_{\rm lgs}\left(\vec{r}_{1}\right) \phi^{T}_{\rm lgs}\left(\vec{r}_{1}\right)\right\rangle + 
\left\langle\phi_{\rm lgs}\left(\vec{r}_{2}\right) \phi^{T}_{\rm lgs}\left(\vec{r}_{2}\right)\right\rangle +
\left\langle\phi_{\rm lgs}\left(\vec{r}_{1}\right) \phi^{T}_{\rm lgs}\left(\vec{r}_{2}\right)\right\rangle +
\left\langle\phi_{\rm lgs}\left(\vec{r}_{2}\right) \phi^{T}_{\rm lgs}\left(\vec{r}_{1}\right)\right\rangle\right]
\end{aligned}
\end{equation}
This equation naturally lends itself to a C++ programming interface in
which a phase covariance function such as $\left\langle\phi^{T}_{\rm
  ttgs}\left(\vec{r}_{1}\right) \phi^{T}_{\rm
  ttgs}\left(\vec{r}_{2}\right)\right\rangle$ is represented as
an instance of a class {\bf phase\_covariance}, with a member function
\\ 
\indent {\bf double phase\_covariance::covariance(vec r\_1, vec r\_2)}
\\
The class instance may store partial calculations specific to the
source location, aperture diameter, and $C_{n}^{2}$ profile.  For
example, the value of $\mathcal{D}_{ab}$ may be computed at
instantiation, while the values of $\mathcal{A}_{ab}\left(\rho\right)$
and $\mathcal{B}_{ab}\left(\rho\right)$ may be computed as a function
of $\rho$ upon first invocation of the above member function.
Likewise, classes {\bf tilt\_covariance} and {\bf phase\_tilt\_covariance}
may represent the other types of covariance functions in Equation
\ref{strfn_x}.  

%\section{Example: Differential Phase Variance}
%\label{sec:variance}

%\begin{equation}\label{uncomp_variance_eqn}
%\end{equation}

%\begin{equation}\label{uncomp_tr_variance_eqn}
%\end{equation}

%\begin{equation}\label{ngs_variance_eqn}
%\end{equation}

%\begin{equation}\label{ngs_tr_variance_eqn}
%\end{equation}

%\begin{equation}\label{lgs_variance_eqn}
%\end{equation}

%\begin{equation}\label{lgs_tr_variance_eqn}
%\end{equation}


%\section{Example: Structure Functions}
%\label{sec:strfn}



%\section{Example: Point Spread Functions}
%\label{sec:psf}




%\section{IDL Bindings}
%\label{sec:idl}
