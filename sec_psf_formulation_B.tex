The adaptive optics system applies compensation based on these
tip-tilt and laser wavefront measurements using fast steering and
deformable mirrors conjugated to the pupil plane.  Because these
mirrors are conjugated in this way, the correction is uniform
throughout the field.  However, the wavefront phase errors are not
uniform throughout the field due to anisoplanatism \cite{Fried:1982}.
Consequently, residual phase aberrations are present in the measured
science target wavefront, and these errors depend on the location of
the science target with respect to the tip tilt and laser sources.

To quantify this description, we may write the residual wavefront
phase error of the science target as
\begin{eqnarray}\label{eqn:phdiff}
\Delta\phi_{\rm sci}\left(\vec{r}\right) & = & \phi_{\rm sci}\left(\vec{r}\right) - \phi^{T}_{\rm ttgs}\left(\vec{r}\right) - 
\left[\phi_{\rm lgs}\left(\vec{r}\right) - \phi^{T}_{\rm lgs}\left(\vec{r}\right)\right] + \phi_{\rm ao}\left(\vec{r}\right) + 
\phi_{\rm inst,sci}\left(\vec{r}\right) \\
& = & \phi_{\rm apl}\left(\vec{r}\right) + \phi_{\rm ao}\left(\vec{r}\right)  + \phi_{\rm inst,sci}\left(\vec{r}\right) \nonumber
\end{eqnarray}
where $\vec{r}$ is the position in the telescope pupil plane. 
The terms above are defined as: 
\begin{equation}\label{eqn:phidefs}
\begin{aligned}
\phi^{T}_{\rm sci}\left(\vec{r}\right)  \quad & \quad  
{\rm Wavefront\; tilt\; of\; science\; target}  \\
\phi_{\rm sci}\left(\vec{r}\right)  \quad & \quad  
{\rm Piston\text{-}removed\; wavefront\; phase\; of\; science\; target}  \\
\Delta\phi_{\rm sci}\left(\vec{r}\right)  \quad & \quad  
{\rm Residual\; piston\text{-}removed\; wavefront\; phase\; of\; science\; target} \\
\phi^{T}_{\rm ttgs}\left(\vec{r}\right)  \quad & \quad  
{\rm Wavefront\; tilt\; of\; tip\text{-}tilt\; natural\; guide\; star}  \\
\phi_{\rm lgs}\left(\vec{r}\right)  \quad & \quad  
{\rm Piston\text{-}removed\; wavefront\; phase\; of\; laser\; guide\; star}  \\
\phi^{T}_{\rm lgs}\left(\vec{r}\right)  \quad & \quad  
{\rm Wavefront\; tilt\; of\; laser\; guide\; star}  \\
\phi_{\rm ao}\left(\vec{r}\right)  \quad & \quad  
{\rm Residual\; piston\text{-}removed\; wavefront\; phase\; from\; AO\; system}  \\
\phi_{\rm inst,sci}\left(\vec{r}\right)  \quad & \quad  
{\rm Piston\text{-}removed\; wavefront\; phase\; from\; instrument\; in\; science\; target\; direction}  \\
\phi_{\rm apl}\left(\vec{r}\right)  \quad & \quad  \phi_{\rm sci}\left(\vec{r}\right) - \phi^{T}_{\rm ttgs}\left(\vec{r}\right) - 
\left[\phi_{\rm lgs}\left(\vec{r}\right) - \phi^{T}_{\rm lgs}\left(\vec{r}\right)\right] = 
{\rm anisoplanatism}
\end{aligned}
\end{equation}
The piston-removed wavefront phase is
employed in these expressions, as the telescope is not sensitive to
atmospheric piston.  Scintillation is neglected in this analysis, as
it is typically a small effect for nighttime turbulence profiles at good
astronomical sites.

The term $\phi_{\rm ao}\left(\vec{r}\right)$ in Equation
\ref{eqn:phidefs} represents the residual piston-removed wavefront phase
error arising from the inability of the adaptive optics system to
perfectly measure and compensate the wavefront.  This phase error
encompasses hardware effects associated with measuring and applying
the tilt to the fast steering mirror and with measuring and applying
the high order wavefront to the deformable mirror.  Consequently, the
phase error $\phi_{\rm ao}\left(\vec{r}\right)$ encompasses
measurement, fitting, and servo error terms.  This phase error does
not include the effects of angular and focal anisoplanatism, which
are broken out explicitly in the other terms in Equation
\ref{eqn:phidefs}. This term is not field dependent as the deformable
mirror is conjugate to the pupil plane. 

The term $\phi_{\rm inst,sci}\left(\vec{r}\right)$ in Equation
\ref{eqn:phidefs} represents wavefront error introduced by the
instrumentation.  This term is field dependent and is distinct from
the residual wavefront error $\phi_{\rm ao}\left(\vec{r}\right)$.
This term in Equation \ref{eqn:phdiff} is interpreted as the
instrumental wavefront error in the direction of the target.  This
term is assumed to be time-independent, as it arises from static
optical elements in the system that are conjugated to locations other
than the pupil plane.  

Equation \ref{eqn:phdiff} represents the residual wavefront phase error
$\Delta\phi_{\rm sci}\left(\vec{r}\right)$ that would exist at
a particular instant in time.  As wind convects turbulence over the
telescope aperture, the telescope receives a succession of random
realizations of the wavefront phase.  While the adaptive optics system
applies compensation in the short exposure limit, the cameras used to
acquire science images expose for timescales that are long compared to
the temporal decorrelation time of the atmosphere.  
Consequently, predictions for the
long exposure point spread function (PFS) formed via ensemble averaging are
of relevance to observational astronomy
\begin{notes}
[CITE]. Why are we discussing this here? Maybe intro?
\end{notes}

To formulate the long exposure PSF, we start with the
definition of the ensemble averaged phase structure function in the
pupil plane.
\begin{equation}
\label{eqn:strfn_def}
D_{\phi}(\vec{r}_{1},\vec{r}_{2})  = 
\left\langle \left\{\phi\left(\vec{r}_{1}\right) - 
\phi\left(\vec{r}_{2}\right)\right\}^{2}\right\rangle
\end{equation}
This structure function may be formed for any of the quantities
defined in Equation \ref{eqn:phidefs}.  The structure function associated
with the residual phase errors in the science target wavefront may be written
\begin{equation}\label{eqn:lgsstrfn}
D_{\Delta\phi_{\rm sci}}(\vec{r}_{1},\vec{r}_{2}) = 
D_{\phi_{\rm apl}}(\vec{r}_{1},\vec{r}_{2}) 
+ D_{\phi_{\rm ao}}(\vec{r}_{1},\vec{r}_{2}) 
+ D_{\phi_{\rm inst,sci}}(\vec{r}_{1},\vec{r}_{2}) 
\end{equation}
This equation assumes that the anisoplanatic errors $\phi_{\rm
  apl}\left(\vec{r}\right)$ and residual errors from the
adaptive optics system $\phi_{\rm ao}\left(\vec{r}\right)$ are
statistically uncorrelated
\begin{notes}
[Do we need to support this assumption].
\end{notes}
The instrumental errors, 
$\phi_{\rm  inst,sci}\left(\vec{r}\right)$ 
are assumed to be static, so that they factor out of the ensemble average.

The long exposure optical transfer function (OTF) of the science target,
${\rm OTF}_{\rm sci}(\vec{r})$, for a circular aperture, may be written 
in terms of the exponentiated structure function as
\begin{equation}\label{eqn:lgsotf}
{\rm OTF}_{\rm sci}(\vec{r}) = 
\int 
W\left(\frac{\vec{s}}{R}\right) \;
W\left(\frac{\vec{r} + \vec{s}}{R}\right)
\exp{ \left\{ -\frac{1}{2} 
D_{\Delta\phi_{\rm sci}}(\vec{s}, \vec{r} + \vec{s})
\right\} } 
\; d\vec{s}
\end{equation}
where the integration is carried out over the pupil plane, $R$ is the
radius of the telescope aperture, and the pupil function
$W\left(\vec{\rho}\right)$ is defined as
\begin{equation}\label{pupileqn}
W\left(\vec{\rho}\right) = 
\left\{
\begin{aligned}
1 & \quad \left\vert \rho \right\vert \le 1 \\
0 & \quad \left\vert \rho \right\vert > 1 \\
\end{aligned}
\right.
\end{equation}
Finally, the long exposure PSF for the science target may be
formed via Fourier transformation of the long exposure optical
transfer function ${\rm OTF}_{\rm sci}(\vec{r})$.

Computing the optical transfer function in Equation \ref{eqn:lgsotf}
requires evaluation of the structure functions 
$D_{\rm \phi_{apl}}(\vec{s},\vec{r} + \vec{s})$, 
$D_{\rm \phi_{ao}}(\vec{s},\vec{r} + \vec{s})$ and
$D_{\rm \phi_{inst,sci}}(\vec{s},\vec{r} + \vec{s})$ as
a function of the pupil plane coordinates $\vec{r}$ and
$\vec{s}$.  In the limiting case where a natural
guide star is used for both tip tilt and high order phase
measurements, the anisoplanatic structure function was shown to be 
stationary \cite{Britton:2006}.
\begin{equation}
D^{\rm NGS}_{\rm apl}(\vec{r}_{1},\vec{r}_{2}) = 
\bar{D}^{\rm NGS}_{\rm apl}(\vec{r}_{1} - \vec{r}_{2}) =
\left\langle \left\{\phi\left(\vec{r}_{1} - \vec{r}_{2}\right) - 
\phi\left(0\right)\right\}^{2}\right\rangle
\end{equation}
In this limiting case, the optical transfer function may be written 
\begin{equation}\label{eqn:ngsotf}
\begin{aligned}
{\rm OTF}^{\rm NGS}_{\rm sci}(\vec{r}) = 
& \exp{ \left\{ -\frac{1}{2} \bar{D}^{\rm NGS}_{\rm apl}(\vec{r}) \right\} }  \\
& \int 
W \left( \frac{\vec{s}}{R} \right)
W \left( \frac{\vec{r}+\vec{s}}{R} \right) 
\exp{ \left\{ -\frac{1}{2} 
\left[
D_{\rm \phi_{ao}}(\vec{s}, \vec{r} + \vec{s}) + 
D_{\rm \phi_{inst,sci}}(\vec{s}, \vec{r} + \vec{s}) 
\right] \right\} }
\; d\vec{s}
\end{aligned}
\end{equation}
This property allows us to factor out the anisoplanatic structure function
from the remaining structure function terms in the calculation of the
OTF.  If one assumes that the differential
instrumental aberrations between the natural guide star $\phi_{\rm inst,
  ngs}\left(\vec{r}\right)$ and the science target $\phi_{\rm
  inst,sci}\left(\vec{r}\right)$ are small, one may apply the
model
\begin{equation}\label{eqn:ngsapproxotf}
\begin{aligned}
{\rm OTF}^{\rm NGS}_{\rm sci}(\vec{r}) \approx
& \exp{ \left\{ -\frac{1}{2} \bar{D}^{\rm NGS}_{\rm apl}(\vec{r}) \right\} }  \\
& \int 
W \left( \frac{\vec{s}}{R} \right)
W \left( \frac{\vec{r} + \vec{s}}{R} \right) 
\exp{ \left\{ -\frac{1}{2} \left[
D_{\rm \phi_{ao}}(\vec{s}, \vec{r} + \vec{s}) + 
D_{\rm \phi_{inst,ngs}}(\vec{s}, \vec{r} + \vec{s})
\right] \right\} }
\; d\vec{s}
\end{aligned}
\end{equation}
The term in the integrand on the right hand side of this equation is
simply the natural guide star OTF, which may be measured
directly from an observation.  This may be combined with a model of
$\bar{D}^{\rm NGS}_{\rm apl}(\vec{r}_{1} - \vec{r}_{2})$ to
predict the off-axis PSF.  Such an experiment was
conducted on the Hale 5m at Palomar Observatory
\cite{Britton:2006}, and demonstrated predictions of the
companion Strehl ratio accurate to 1\% at both $\lambda=1.65\;\mu$m and
$\lambda=2.12\;\mu$m out to separations of $\sim$12''.
During that experiment, the
assumption that $\phi_{\rm inst,ngs}\left(\vec{r}\right) \approx
\phi_{\rm inst,sci}\left(\vec{r}\right)$ did not significantly
impact the accuracy of the prediction of ${\rm OTF}^{\rm NGS}_{\rm
  sci}(\vec{r})$. 

Unfortunately, the anisoplanatic structure function in Equation
\ref{eqn:lgsstrfn} does not manifest the property of stationarity for LGS-AO.  
Consequently, the OTF cannot be factored as shown in Equation
\ref{eqn:ngsotf}.  Despite this complication, there are approaches that
will advance our understanding of the point spread function delivered
by a LGS-AO system.  First, one may assume
perfect adaptive optics compensation and no instrumental error, so
that $\phi_{\rm ao}\left(\vec{r}\right) = 
\phi_{\rm inst,sci}\left(\vec{r}\right) = 0$.  This assumption permits
calculation of the impact of anisoplanatism on the OTF 
and hence on the PSF. While this is not the approach we take in 
AIROPA, this simplified case
can provide significant insight, as the PSF of the science target depends
strongly on the source geometry, aperture diameter, turbulence
profile, and zenith angle.  Under this assumption
\begin{equation}\label{eqn:lgsotfa}
\begin{aligned}
{\rm OTF}^{\rm LGS}_{\rm sci}(\vec{r}) = 
\int 
W \left( \frac{\vec{s}}{R} \right)
W \left( \frac{\vec{r} + \vec{s}}{R} \right) 
\exp{\left\{ -\frac{1}{2} \left[
D_{\rm \phi_{apl}}(\vec{s}, \vec{r} + \vec{s})
\right] \right\}}
\; d\vec{s} 
\end{aligned}
\end{equation}
As an illustration, a PSF and OTF are shown in Figure
\ref{fig:ngs_perfect} for both an on-axis and off-axis (5'') science
target simulated for the Keck 2 adaptive optics system in LGS mode
with both the LGS and the TTS located at the same on-axis position. 
Only the atmosphere and the telescope pupil are included as described
by Equation \ref{eqn:lgsotfa}. 