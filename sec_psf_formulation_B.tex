Light from sources, such as those shown in Figure \ref{fig:guide_star_schematic} descend
through columns of atmosphere to arrive at the aperture of the
telescope.  During their passage through the atmosphere, wavefronts
acquire phase aberrations due to index of refraction fluctuations in
air arising from turbulence 
\begin{notes}
[CITE].
\end{notes}
These wavefronts arrive at the
telescope, where they are sensed by detectors.  The wavefront from the
tip tilt guide star is sensed by a tracker to determine wavefront
tilt, while that from the laser is sensed by a wavefront sensor to
determine high order wavefront aberrations.  

The adaptive optics system applies compensation based on these tip
tilt and laser wavefront measurements using fast steering and
deformable mirrors conjugated to the pupil plane.  Because these
mirrors are conjugated in this way, the correction is uniform
throughout the field.  However, the wavefront phase errors are not
uniform throughout the field due to anisoplanatism 
\begin{notes}
[CITE].
\end{notes} 
Consequently, residual phase aberrations are present in the measured target
wavefront, and these errors depend on the location of the target with
respect to the tip tilt and laser sources.

To quantify this description, we may write the residual wavefront
phase error of the target as
\begin{eqnarray}\label{eqn:phdiff}
\Delta\phi_{\rm sci}\left(\boldsymbol{r}\right) & = & \phi_{\rm sci}\left(\boldsymbol{r}\right) - \phi^{T}_{\rm ttgs}\left(\boldsymbol{r}\right) - 
\left[\phi_{\rm lgs}\left(\boldsymbol{r}\right) - \phi^{T}_{\rm lgs}\left(\boldsymbol{r}\right)\right] + \phi_{\rm ao}\left(\boldsymbol{r}\right) + 
\phi_{\rm inst,sci}\left(\boldsymbol{r}\right) \\
& = & \phi_{\rm apl}\left(\boldsymbol{r}\right) + \phi_{\rm ao}\left(\boldsymbol{r}\right)  + \phi_{\rm inst,sci}\left(\boldsymbol{r}\right) \nonumber
\end{eqnarray}
where $\boldsymbol{r}$ is the position in the telescope pupil plane. 
The terms above are defined as: 
\begin{notes}
Fix the ``piston-removed'' dashes.
\end{notes}
\begin{equation}\label{eqn:phidefs}
\begin{aligned}
\phi^{T}_{\rm sci}\left(\boldsymbol{r}\right)  \quad & \quad  
{\rm Wavefront\; tilt\; of\; science\; target}  \\
\phi_{\rm sci}\left(\boldsymbol{r}\right)  \quad & \quad  
{\rm Piston-removed\; wavefront\; phase\; of\; science\; target}  \\
\Delta\phi_{\rm sci}\left(\boldsymbol{r}\right)  \quad & \quad  
{\rm Residual\; piston-removed\; wavefront\; phase\; of\; science\; target} \\
\phi^{T}_{\rm ttgs}\left(\boldsymbol{r}\right)  \quad & \quad  
{\rm Wavefront\; tilt\; of\; tip-tilt\; natural\; guide\; star}  \\
\phi_{\rm lgs}\left(\boldsymbol{r}\right)  \quad & \quad  
{\rm Piston-removed\; wavefront\; phase\; of\; laser\; guide\; star}  \\
\phi^{T}_{\rm lgs}\left(\boldsymbol{r}\right)  \quad & \quad  
{\rm Wavefront\; tilt\; of\; laser\; guide\; star}  \\
\phi_{\rm ao}\left(\boldsymbol{r}\right)  \quad & \quad  
{\rm Residual\; piston-removed\; wavefront\; phase\; from\; AO\; system}  \\
\phi_{\rm inst,sci}\left(\boldsymbol{r}\right)  \quad & \quad  
{\rm Piston-removed\; wavefront\; phase\; from\; instrument\; in\; science\; target\; direction}  \\
\phi_{\rm apl}\left(\boldsymbol{r}\right)  \quad & \quad  \phi_{\rm sci}\left(\boldsymbol{r}\right) - \phi^{T}_{\rm ttgs}\left(\boldsymbol{r}\right) - 
\left[\phi_{\rm lgs}\left(\boldsymbol{r}\right) - \phi^{T}_{\rm lgs}\left(\boldsymbol{r}\right)\right] = 
{\rm anisoplanatism}
\end{aligned}
\end{equation}
The piston-removed wavefront phase is
employed in these expressions, as the telescope is not sensitive to
atmospheric piston.  Scintillation is neglected in this analysis, as
it is a small effect for nighttime turbulence profiles at good
astronomical sites.

The term $\phi_{\rm ao}\left(\boldsymbol{r}\right)$ in Equation
\ref{eqn:phidefs} represents the residual piston-removed wavefront phase
error arising from the inability of the adaptive optics system to
perfectly measure and compensate the wavefront.  This phase error
encompasses hardware effects associated with measuring and applying
the tilt to the fast steering mirror and with measuring and applying
the high order wavefront to the deformable mirror.  Consequently, the
phase error $\phi_{\rm ao}\left(\boldsymbol{r}\right)$ encompasses
measurement, fitting, and servo error terms.  This phase error does
not include the effects of angular and focal anisoplanatism, which
are broken out explicitly in the other terms in Equation
\ref{eqn:phidefs}.

The term $\phi_{\rm inst,sci}\left(\boldsymbol{r}\right)$ in Equation
\ref{eqn:phidefs} represents wavefront error introduced by the
instrumentation.  This term is field dependent, and is distinct from
the residual wavefront error $\phi_{\rm
  ao}\left(\boldsymbol{r}\right)$ that arises from the deformable
mirror conjugated to the pupil plane, which is field independent.
Consequently, the term in Equation \ref{eqn:phdiff} is interpreted as the
instrumental wavefront error in the direction of the target.  This
term is assumed to be time-independent, as it arises from static
optical elements in the system that are conjugated to locations other
than the pupil plane.  

Equation \ref{eqn:phdiff} represents the residual wavefront phase error
$\Delta\phi_{\rm sci}\left(\boldsymbol{r}\right)$ that would exist at
a particular instant in time.  As wind convects turbulence over the
telescope aperture, the telescope receives a succession of random
realizations of the wavefront phase.  While the adaptive optics system
applies compensation in the short exposure limit, the cameras used to
acquire science images expose for timescales that are long compared to
the temporal decorrelation time of the atmosphere.  
Consequently, predictions for the
long exposure point spread function (PFS) formed via ensemble averaging are
of relevance to observational astronomy
\begin{notes}
[CITE].
\end{notes}

To formulate the long exposure PSF, we start with the
definition of the ensemble averaged phase structure function in the
pupil plane.
\begin{equation}
\label{eqn:strfn_def}
D_{\phi}(\boldsymbol{r}_{1},\boldsymbol{r}_{2})  = 
\left\langle \left\{\phi\left(\boldsymbol{r}_{1}\right) - 
\phi\left(\boldsymbol{r}_{2}\right)\right\}^{2}\right\rangle
\end{equation}
This structure function may be formed for any of the quantities
defined in Equation \ref{eqn:phidefs}.  The structure function associated
with the residual phase errors in the science target wavefront may be written
\begin{equation}\label{eqn:lgsstrfn}
D_{\Delta\phi_{\rm sci}}(\boldsymbol{r}_{1},\boldsymbol{r}_{2}) = 
D_{\phi_{\rm apl}}(\boldsymbol{r}_{1},\boldsymbol{r}_{2}) 
+ D_{\phi_{\rm ao}}(\boldsymbol{r}_{1},\boldsymbol{r}_{2}) 
+ D_{\phi_{\rm inst,sci}}(\boldsymbol{r}_{1},\boldsymbol{r}_{2}) 
\end{equation}
This equation assumes that the anisoplanatic errors $\phi_{\rm
  apl}\left(\boldsymbol{r}\right)$ and residual errors from the
adaptive optics system $\phi_{\rm ao}\left(\boldsymbol{r}\right)$ are
statistically uncorrelated
\begin{notes}
[Do we need to support this assumption].
\end{notes}
The instrumental errors, 
$\phi_{\rm  inst,sci}\left(\boldsymbol{r}\right)$ 
are assumed to be static, so that they factor out of the ensemble average.

The long exposure optical transfer function (OTF) of the science target,
${\rm OTF}_{\rm sci}(\boldsymbol{r})$, for a circular aperture, may be written 
in terms of the exponentiated structure function as
\begin{equation}\label{eqn:lgsotf}
{\rm OTF}_{\rm sci}(\boldsymbol{r}) = 
\int 
W\left(\frac{\boldsymbol{s}}{R}\right) \;
W\left(\frac{\boldsymbol{r} + \boldsymbol{s}}{R}\right)
\exp{ \left\{ -\frac{1}{2} 
D_{\Delta\phi_{\rm sci}}(\boldsymbol{s}, \boldsymbol{r} + \boldsymbol{s})
\right\} } 
\; d\boldsymbol{s}
\end{equation}
where the integration is carried out over the pupil plane, $R$ is the
radius of the telescope aperture, and the pupil function
$W\left(\boldsymbol{\rho}\right)$ is defined as
\begin{equation}\label{pupileqn}
W\left(\boldsymbol{\rho}\right) = 
\left\{
\begin{aligned}
1 & \quad \left\vert \rho \right\vert \le 1 \\
0 & \quad \left\vert \rho \right\vert > 1 \\
\end{aligned}
\right.
\end{equation}
Finally, the long exposure PSF for the science target may be
formed via Fourier transformation of the long exposure optical
transfer function ${\rm OTF}_{\rm sci}(\boldsymbol{r})$.

Computing the optical transfer function in Equation \ref{eqn:lgsotf}
requires evaluation of the structure functions $D_{\rm
  apl}(\boldsymbol{s},\boldsymbol{r} + \boldsymbol{s})$, $D_{\rm
  \phi_{ao}}(\boldsymbol{s},\boldsymbol{r} + \boldsymbol{s})$ and
$D_{\rm \phi_{inst,sci}}(\boldsymbol{s},\boldsymbol{r} + \boldsymbol{s})$ as
a function of the pupil plane coordinates $\boldsymbol{r}$ and
$\boldsymbol{s}$.  In the limiting case where the same natural
guide star is used for both tip tilt and high order phase
measurements, the anisoplanatic structure function was shown to be 
stationary \cite{Britton:2006}.
\begin{equation}
D^{\rm NGS}_{\rm apl}(\boldsymbol{r}_{1},\boldsymbol{r}_{2}) = 
D^{\rm NGS}_{\rm apl}(\boldsymbol{r}_{1} - \boldsymbol{r}_{2}) 
\end{equation}
In this limiting case, the optical transfer function may be written 
\begin{equation}\label{eqn:ngsotf}
\begin{aligned}
{\rm OTF}^{\rm NGS}_{\rm sci}(\boldsymbol{r}) = 
& \exp{ \left\{ -\frac{1}{2}  D^{\rm NGS}_{\rm apl}(\boldsymbol{r}) \right\} }  \\
& \int 
W \left( \frac{\boldsymbol{s}}{R} \right)
W \left( \frac{\boldsymbol{r}+\boldsymbol{s}}{R} \right) 
\exp{ \left\{ -\frac{1}{2} 
\left[
D_{\rm \phi_{ao}}(\boldsymbol{s}, \boldsymbol{r} + \boldsymbol{s}) + 
D_{\rm \phi_{inst,sci}}(\boldsymbol{s}, \boldsymbol{r} + \boldsymbol{s}) 
\right] \right\} }
\; d\boldsymbol{s}
\end{aligned}
\end{equation}
This property allows us to factor out the anisoplanatic structure function
from the remaining structure function terms in the calculation of the
OTF.  If one assumes that the differential
instrumental aberrations between the natural guide star $\phi_{\rm inst,
  ngs}\left(\boldsymbol{r}\right)$ and the science target $\phi_{\rm
  inst,sci}\left(\boldsymbol{r}\right)$ are small, one may apply the
model
\begin{equation}\label{eqn:ngsapproxotf}
\begin{aligned}
{\rm OTF}^{\rm NGS}_{\rm sci}(\boldsymbol{r}) \approx
& \exp{ \left\{ -\frac{1}{2} D^{\rm NGS}_{\rm apl}(\boldsymbol{r}) \right\} }  \\
& \int 
W \left( \frac{\boldsymbol{s}}{R} \right)
W \left( \frac{\boldsymbol{r} + \boldsymbol{s}}{R} \right) 
\exp{ \left\{ -\frac{1}{2} \left[
D_{\rm \phi_{ao}}(\boldsymbol{s}, \boldsymbol{r} + \boldsymbol{s}) + 
D_{\rm \phi_{inst,ngs}}(\boldsymbol{s}, \boldsymbol{r} + \boldsymbol{s})
\right] \right\} }
\; d\boldsymbol{s}
\end{aligned}
\end{equation}
The term in the integrand on the right hand side of this equation is
simply the natural guide star OTF, which may be measured
directly from an observation.  This may be combined with a model of
$D^{\rm NGS}_{\rm apl}(\boldsymbol{r}_{1} - \boldsymbol{r}_{2})$ to
predict the off-axis PSF.  Such an experiment was
conducted on the Hale 5m at Palomar Observatory
\cite{Britton:2006}, and demonstrated predictions of the
companion Strehl ratio accurate to 1\% at both $\lambda=1.65\;\mu$m and
$\lambda=2.12\;\mu$m
\begin{notes}
out to separations of YYY.
\end{notes}
For the particular telescope, natural guide star adaptive optics
system, and near-infrared camera employed in this experiment, the
assumption $\phi_{\rm inst,ngs}\left(\boldsymbol{r}\right) \approx
\phi_{\rm inst,sci}\left(\boldsymbol{r}\right)$ did not significantly
impact the accuracy of the prediction of ${\rm OTF}^{\rm NGS}_{\rm
  sci}(\boldsymbol{r})$.

Unfortunately, the anisoplanatic structure function in Equation
\ref{eqn:lgsstrfn} does not manifest the property of stationarity for LGS-AO.  
Consequently, the OTF cannot be factored as shown in Equation
\ref{eqn:ngsotf}.  Despite this complication, there are approaches that
will advance our understanding of the point spread function delivered
by a LGS-AO system.  First, one may assume
perfect adaptive optics compensation and no instrumental error, so
that $\phi_{\rm ao}\left(\boldsymbol{r}\right) = 
\phi_{\rm inst,sci}\left(\boldsymbol{r}\right) = 0$.  This assumption permits
calculation of the impact of anisoplanatism on the OTF 
and hence on the PSF. This simplified case
can provide significant insight, as the PSF of the science target depends
strongly on the source geometry, aperture diameter, turbulence
profile, and zenith angle.  Under this assumption
\begin{equation}\label{eqn:lgsotfa}
\begin{aligned}
{\rm OTF}^{\rm LGS}_{\rm sci}(\boldsymbol{r}) = 
\int 
W \left( \frac{\boldsymbol{s}}{R} \right)
W \left( \frac{\boldsymbol{r} + \boldsymbol{s}}{R} \right) 
\exp{\left\{ -\frac{1}{2} \left[
D_{\rm \phi_{apl}}(\boldsymbol{s}, \boldsymbol{r} + \boldsymbol{s})
\right] \right\}}
\; d\boldsymbol{s} 
\end{aligned}
\end{equation}
\begin{notes}
SAY whether we adopt this approach.
\end{notes}

A second possible approach lies in measuring the structure functions
$D_{\rm \phi_{inst,sci}}(\boldsymbol{s}, \boldsymbol{r} +\boldsymbol{s} )$ 
and
$D_{\rm \phi_{ao}}(\boldsymbol{s}, \boldsymbol{r} + \boldsymbol{s})$
and using the measured values directly in Equation
\ref{eqn:lgsotf}.  The static, field-dependent instrumental wavefront
error, $\phi_{\rm inst,sci}(\boldsymbol{r})$, may be measured using a technique 
such as phase diversity.  The
structure function may then be formed by brute force numerical
calculation via
\begin{equation}
\label{eqn:strfn_inst}
D_{\rm \phi_{inst,sci}}(\boldsymbol{r}_{1}, \boldsymbol{r}_{2})  = 
\left\langle \left\{\phi_{\rm inst,sci}\left(\boldsymbol{r}_{1}\right) - 
\phi_{\rm inst,sci}\left(\boldsymbol{r}_{2}\right)\right\}^{2}\right\rangle
\end{equation}
Similarly, the statistical properties of $\phi_{\rm
  ao}\left(\boldsymbol{r}\right)$ may be characterized via modelling
of the adaptive optics system.  Such a characterization would involve
understanding the contribution to the structure function $D_{\rm
  \phi_{ao}}(\boldsymbol{r}_{1},\boldsymbol{r}_{2})$ from measurement errors,
servo and fitting errors for specific hardware and under specific
observing conditions.  For example, statistical properties of 
high-order fitting error are determined by the turbulence profile and
actuator pitch on the deformable mirror
\begin{notes}
[CITE].
\end{notes}
Similarly, statistical
properties of high-order measurement error are dictated by the
geometry of the wavefront sensor, noise and diffusion properties of
the wavefront sensor detector, and by the guide star brightness.  Finally,
the statistical properties of high-order servo errors are dictated by the
turbulence and wind profiles and by the latency in the real time
controller.  If the four dimensional structure function $D_{\rm
  \phi_{ao}}(\boldsymbol{r}_{1},\boldsymbol{r}_{2})$ may be modelled
successfully, then the OTF may be computed
directly from Equation \ref{eqn:lgsotf}.  Modelling of $D_{\rm
  \phi_{ao}}(\boldsymbol{r}_{1},\boldsymbol{r}_{2})$ at Keck is beyond 
the scope of this project and is being pursued in a parallel effort funded 
through the NSF ATI program 
\begin{notes}
[PI: Wizinowich, NSF-ATI number].
\end{notes}
The results of this modelling may demonstrate that 
$D_{\rm \phi_{ao}}(\boldsymbol{r}_{1},\boldsymbol{r}_{2})$ is nearly
stationary.  This might be argued based on the fact that fitting,
servo and measurement errors tend to decorrelate at separations of a
single subaperture, and do so uniformly over the pupil plane.  This
leads to structure functions that are stationary.  Were this
assumption to hold, one could write Equation \ref{eqn:lgsotf} as
\begin{equation}\label{eqn:lgsotfb}
\begin{aligned}
{\rm OTF}^{\rm LGS}_{\rm sci}(\boldsymbol{r}) = 
& \exp{
\left\{ -\frac{1}{2} D_{\rm \phi_{\rm ao}}(\boldsymbol{r}) \right\}}  \\
& \int 
W \left( \frac{\boldsymbol{s}}{R} \right)
W \left( \frac{\boldsymbol{r} + \boldsymbol{s}}{R} \right)
\exp{ \left\{ -\frac{1}{2} \left[
D_{\rm \phi_{apl}}(\boldsymbol{s}, \boldsymbol{r} + \boldsymbol{s}) +
D_{\rm \phi_{inst,sci}}(\boldsymbol{s}, \boldsymbol{r} + \boldsymbol{s})
\right] \right\} }
\; d\boldsymbol{s} 
\end{aligned}
\end{equation}
Naturally this stationarity assumption must be validated against
performance of real hardware, and its validity is dictated by the
modelling accuracy required for the astronomical application.  
\begin{notes}
ARE we adopting this assumption? 
\end{notes}

Finally, one could postulate a model that assumes the instrumental
structure function is stationary, so that
\begin{equation}\label{eqn:lgsotfc}
\begin{aligned}
{\rm OTF}^{\rm LGS}_{\rm sci}(\boldsymbol{r}) = 
& \exp{ \left\{ -\frac{1}{2}
\left[ D_{\rm \phi_{\rm ao}}(\boldsymbol{r}) \right] \right\} } 
\exp{ \left\{ -\frac{1}{2}
\left[ D_{\rm \phi_{\rm inst,sci}}(\boldsymbol{r}) \right] \right\} }  \\
& \int 
W \left( \frac{\boldsymbol{s}}{R} \right)
W \left( \frac{\boldsymbol{r} + \boldsymbol{s}}{R} \right)
\exp{ \left\{ -\frac{1}{2} \left[
D_{\rm \phi_{apl}}(\boldsymbol{s},\boldsymbol{r} + \boldsymbol{s})
\right] \right\} }
\; d\boldsymbol{s} 
\end{aligned}
\end{equation}
This approximation would be suitable when 
$\phi_{\rm inst,sci} \left( \boldsymbol{r} \right) \ll 
\phi_{\rm aniso} \left( \boldsymbol{r} \right)$.

The validity of the approximations in Equations \ref{eqn:lgsotfa},
\ref{eqn:lgsotfb} and \ref{eqn:lgsotfc} to the model in Equation \ref{eqn:lgsotf}
will depend signficantly on the character of the adaptive optics
system, the optical properties of the instrumentation, and the
importance of these terms relative to the degree of anisoplanatism
inherent in the observational scenario.  The importance of
anisoplanatism itself depends on many factors: the distribution of the
sources with respect to the laser and tip tilt guide star, the
turbulence profile, zenith angle, observing wavelength, and aperture
diameter.  The state of our knowledge regarding $D_{\rm
  \phi_{ao}}(\boldsymbol{s},\boldsymbol{r} + \boldsymbol{s})$ and
$D_{\rm \phi_{inst,sci}}(\boldsymbol{s},\boldsymbol{r} +
\boldsymbol{s})$ at Keck Observatory is currently evolving.  This
report presents the methodology to compute $D_{\rm
  \phi_{aniso}}(\boldsymbol{s},\boldsymbol{r})$.  Knowledge of each of
these quantities is necessary to formulate the prediction of
${\rm OTF}^{\rm LGS}_{\rm sci}(\boldsymbol{r})$ in Equation \ref{eqn:lgsotf}, or to
utilize the approximations in Equations \ref{eqn:lgsotfa}, \ref{eqn:lgsotfb}
and \ref{eqn:lgsotfc}.

Proceeding along this line of inquiry, the anisoplanatic structure
function may be expressed in terms of the quantities defined in
Equation \ref{eqn:phidefs} as
\begin{equation}\label{eqn:lgsstrfnexpansion}
\begin{split}
D_{\phi_{\rm apl}}(\boldsymbol{r}_{1},\boldsymbol{r}_{2}) & =
\left\langle \left\{\phi_{\rm apl}\left(\boldsymbol{r}_{1}\right) - 
\phi_{\rm apl}\left(\boldsymbol{r}_{2}\right)\right\}^{2}\right\rangle \\
& 
=\left\langle \left\{
\left[\phi_{\rm sci}\left(\boldsymbol{r}_{1}\right) - \phi^{T}_{\rm ttgs}\left(\boldsymbol{r}_{1}\right) - 
\phi_{\rm lgs}\left(\boldsymbol{r}_{1}\right) + \phi^{T}_{\rm lgs}\left(\boldsymbol{r}_{1}\right)\right] -
\vphantom{\left\langle\left\{\right\}^{2}\right\rangle}
\right.
\right.
\\
& \quad
\left.
\left.
\left[\phi_{\rm sci}\left(\boldsymbol{r}_{2}\right) - \phi^{T}_{\rm ttgs}\left(\boldsymbol{r}_{2}\right) - 
\phi_{\rm lgs}\left(\boldsymbol{r}_{2}\right) + \phi^{T}_{\rm lgs}\left(\boldsymbol{r}_{2}\right)\right]
\right\}^{2}\right\rangle  
\end{split}
\end{equation}
This expression may be written as the sum of 36 two-point phase
covariance functions.  For example, the term $\left\langle\phi_{\rm
  sci}\left(\boldsymbol{r}_{1}\right)\phi_{\rm
  sci}\left(\boldsymbol{r}_{2}\right)\right\rangle$ represents the
target piston-removed wavefront phase covariance.  For the special
case $\boldsymbol{r}_{1}=\boldsymbol{r}_{2}\equiv\boldsymbol{r}$, this
term simplifies to $\left\langle \left\{\phi_{\rm
  sci}\left(\boldsymbol{r}\right)\right\}^{2}\right\rangle$,
representing the target piston-removed wavefront phase variance as a
function of the pupil plane coordinate $\boldsymbol{r}$.  As a
concrete example of the utility of this function, we could measure the
piston-removed wavefront phase using a Shack Hartmann sensor and
compute the variance at each subaperture in the pupil plane.  The
function $\left\langle\left\{\phi_{\rm
  sci}\left(\boldsymbol{r}\right)^{2}\right\}\right\rangle$ would represent
the Komolgorov prediction corresponding to our measurement.

\section{Two-Point Covariance Functions}
\label{sec:covariance}

The key elements required for the prediction of the structure function
in Equation \ref{eqn:lgsstrfnexpansion} and hence the optical transfer
function in Equation \ref{eqn:lgsotfa} are the two-point covariance
functions that appear in Equation \ref{eqn:lgsstrfnexpansion}.  The
following three such functions are required to compute the structure
function.
\begin{equation}\label{tpcf}
\begin{aligned}
\left\langle \phi_{a}\left(\boldsymbol{r}_{1}\right) \phi_{b}\left(\boldsymbol{r}_{2}\right) \right\rangle &\quad\quad\quad {\rm Piston\;removed\;phase\;covariance} \\
\left\langle \phi^{T}_{a}\left(\boldsymbol{r}_{1}\right) \phi^{T}_{b}\left(\boldsymbol{r}_{2}\right) \right\rangle &\quad\quad\quad {\rm Tilt\;phase\;covariance} \\
\left\langle \phi_{a}\left(\boldsymbol{r}_{1}\right) \phi^{T}_{b}\left(\boldsymbol{r}_{2}\right) \right\rangle &\quad\quad\quad {\rm Piston\;removed\;/\;tilt\;cross\;covariance} \\
\end{aligned}
\end{equation}
In this expression the subscripts $a$ and $b$ label the two sources,
while the vectors $\boldsymbol{r}_{1}$ and $\boldsymbol{r}_{2}$
represent coordinates in the telescope aperture.  This section
sketches the derivation of this quantity assuming a Komolgorov
turbulence spectrum.  This derivation appears in
\cite{Tyler:1994}, and is outlined here for
completeness.  A complementary approach appears in 
\cite{Sasiela:2007}.
\begin{notes}
Do we really need to re-hash all of this? 
\end{notes}

Wavefront phase aberrations arise from the passage of light through
the atmosphere as this light encounters variations in the index of
refraction $n\left(\boldsymbol{r},z\right)$.  Consider a geometric ray
traversing the path between a source at range $\mathcal{R}_{a}$ and
lateral offset $\boldsymbol{r}_{a}$ and a location in the pupil plane
denoted by $\boldsymbol{r}$, as shown in Figure
\ref{eqn:indexofrefraction}.  The total phase aberration is given by the integral
along the geometric ray.
\begin{equation}\label{eqn:totalphase_n}
\hat{\phi}_{a}\left(\boldsymbol{r}\right) = k \int_{0}^{\mathcal{R}_{a}} dz \; 
n\left[\boldsymbol{r}\left( \frac{1-z}{\mathcal{R}_{a}} \right) + \boldsymbol{r}_{a} \frac{z}{\mathcal{R}_{a}}, z\right]
\end{equation}


%  \myFigureI {indexofrefraction}
% {{figs/indexofrefraction/indexofrefraction}{}} 
% {{6.5cm}{.6\textwidth}}
% { {Geometric Raytrace} {A ray emitted by a source at range
%     $\mathcal{R}_{a}$ and lateral offset $\boldsymbol{r}_{a}$ descends
%     through turbulence to arrive at a location $\boldsymbol{r}$ in the
%     telescope pupil plane.  The wavefront phase aberration at location
%     $\boldsymbol{r}$ is proportional to the integral of the index of
%     refraction $n\left[\boldsymbol{r}\left(1-z/\mathcal{R}_{a}\right) +
%       \boldsymbol{r}_{a} z/\mathcal{R}_{a}, z\right]$ along the geometric
%     path.}}


The aperture averaged piston component of the wavefront phase is given by the average of
Equation \ref{eqn:totalphase_n} over the telescope aperture.
\begin{equation}\label{eqn:pistonphase_n}
\bar{\phi}_{a} = \frac{k}{\pi R^{2}} \int_{0}^{\mathcal{R}_{a}} dz 
\int d\boldsymbol{r}^{\prime} \; 
W \left( \frac{\boldsymbol{r}^{\prime}}{R} \right) \;
n
\left[\boldsymbol{r}^{\prime} \left( \frac{1-z}{\mathcal{R}_{a}} \right) + 
\boldsymbol{r}_{a} \frac{z}{\mathcal{R}_{a}}, z \right]
\end{equation}
The piston removed component of the wavefront phase is then
\begin{equation}\label{prphase_n}
\phi_{a}\left(\boldsymbol{r}\right) = \hat{\phi}_{a}\left(\boldsymbol{r}\right) - \bar{\phi}_{a}
\end{equation}
The tilt is given by the coordinate weighted average of Equation
\ref{totalphase_n} over the telescope aperture.
\begin{equation}\label{tilt_n}
\begin{aligned}
\boldsymbol{\theta} &= \frac{4}{\pi R^{4}} \int_{0}^{\mathcal{R}_{a}} dz 
\int d\boldsymbol{r}^{\prime} \; \boldsymbol{r}^{\prime} \;
W \left( \frac{\boldsymbol{r}^{\prime}}{R} \right) \;
n
\left[ \boldsymbol{r}^{\prime} \left(\frac{1-z}{\mathcal{R}_{a}}\right) + 
\boldsymbol{r}_{a} \frac{z}{\mathcal{R}_{a}}, z\right]
\end{aligned}
\end{equation}
Definition of this quantity is often confusing.  There is a
distinction between angular tilt, tilt optical path difference, and
tilt phase.  The first quantity is the angle formed by the physical,
tilted surface with respect to the aperture plane.  This is the
quantity $\boldsymbol{\theta}$ expressed in Equation \ref{tilt_n}.
The second is the physical distance between the tilted surface and the
aperture plane.  The final quantity is the physical difference
multiplied by the wavenumber $k$, and measures tilt phase.  As a
second point, tilt variances may be quoted as one-axis one-sigma or as
two-axis one-sigma.  This report employs the one-axis one-sigma
convention.  To define terms and demonstrate that the coefficient in
Equation \ref{tilt_n} is correct, consider a hypothetical index of
refraction function $n\left[\boldsymbol{r},z\right] = \alpha r \cos
\theta \delta\left(z\right)$, where $\delta\left(z\right)$ is the
Kronecker delta function.  Equation \ref{tilt_n} should measure the
angular tilt coefficient $\alpha$, so that $\boldsymbol{\theta} =
\left(\theta_{x}, \theta_{y}\right) = \left(\alpha, 0\right)$.  We
have
\begin{equation}
\begin{aligned}
\boldsymbol{\theta} & = 
\left\{
\begin{array}{c}
\theta_{x} \\
\theta_{y} 
\end{array} 
\right\}\\
& = 
\frac{4}{\pi R^{4}} \int_{0}^{\mathcal{R}_{a}} dz 
\int d\boldsymbol{r}^{\prime} \boldsymbol{r}^{\prime} W(\boldsymbol{r}^{\prime}/R) \; n\left[\boldsymbol{r},z\right] \\
& = 
\frac{4}{\pi R^{4}} \int_{0}^{\mathcal{R}_{a}} dz \delta\left(z\right)
\int_{0}^{R} r dr \int_{0}^{2\pi} d\theta 
\left\{
\begin{array}{c}
r \cos \theta \\
r \sin \theta \\ 
\end{array}
\right\}
\left\{
\begin{array}{c}
\alpha r \cos \theta \\
0 \\ 
\end{array}
\right\} \\
& = 
\frac{4}{\pi R^{4}}  
\int_{0}^{R} r dr \int_{0}^{2\pi} d\theta 
\left\{
\begin{array}{c}
\alpha r^{2} \cos^{2} \theta \\
0 \\ 
\end{array}
\right\}\\
& = 
\left\{
\begin{array}{c}
\alpha \\
0 \\ 
\end{array}
\right\}
\end{aligned}
\end{equation}
as anticipated.  By simple geometry, the tilt phase is the dot product
of the tilt $\boldsymbol{\theta}$ with the coordinate offset
$\boldsymbol{r}$ multiplied by the wavenumber.
\begin{equation}\label{tiltphase_n}
\phi^{T}_{a}\left(\boldsymbol{r}\right) = k \boldsymbol{r}  \cdot \boldsymbol{\theta} 
\end{equation}

Equations \ref{totalphase_n} - \ref{tilt_n} and Equation
\ref{tiltphase_n} are the expressions required to form the covariance
functions in Equation \ref{tpcf}.  The two-point piston removed phase
covariance function is
\begin{equation}\label{prphase_n}
\begin{aligned}
&\left\langle \vphantom{\hat{\phi}_{a}} \phi_{a}\left(\boldsymbol{r}_{1}\right) \phi_{b}\left(\boldsymbol{r}_{2}\right)\right\rangle = \\
&\quad\quad\quad\quad
\frac{k^{2}}{\pi^{2}R^{4}} \int_{0}^{\mathcal{R}_{a}}  \int_{0}^{\mathcal{R}_{b}} \iint dz_{1} \; dz_{2} \; d\boldsymbol{r}^{\prime}_{1}d\boldsymbol{r}^{\prime}_{2} W(\boldsymbol{r}^{\prime}_{1}/R) W(\boldsymbol{r}^{\prime}_{2}/R) \\
&\quad\quad\quad\quad\quad\quad
\left\{\vphantom{\boldsymbol{r}^{\prime}_{1}}
\left\langle n\left[\boldsymbol{r}_{1}\left(1-z_{1}/\mathcal{R}_{a}\right) + \boldsymbol{r}_{a} z_{1}/\mathcal{R}_{a}, z_{1}\right]
 n\left[\boldsymbol{r}_{2}\left(1-z_{2}/\mathcal{R}_{b}\right) + \boldsymbol{r}_{b} z_{2}/\mathcal{R}_{b}, z_{2}\right] \right\rangle\right. -
\\
&\quad\quad\quad\quad\quad\quad
\left\langle n\left[\boldsymbol{r}^{\prime}_{1}\left(1-z_{1}/\mathcal{R}_{a}\right) + \boldsymbol{r}_{a} z_{1}/\mathcal{R}_{a}, z_{2}\right]
 n\left[\boldsymbol{r}_{2}\left(1-z_{2}/\mathcal{R}_{b}\right) + \boldsymbol{r}_{b} z_{2}/\mathcal{R}_{b}, z_{2}\right] \right\rangle -
\\
&\quad\quad\quad\quad\quad\quad
\left\langle n\left[\boldsymbol{r}_{1}\left(1-z_{1}/\mathcal{R}_{a}\right) + \boldsymbol{r}_{a} z_{1}/\mathcal{R}_{a}, z_{1}\right]
 n\left[\boldsymbol{r}^{\prime}_{2}\left(1-z_{2}/\mathcal{R}_{b}\right) + \boldsymbol{r}_{b} z_{2}/\mathcal{R}_{b}, z_{2}\right] \right\rangle +
\\
&\quad\quad\quad\quad\quad\quad
\left.
\left\langle n\left[\boldsymbol{r}^{\prime}_{1}\left(1-z_{1}/\mathcal{R}_{a}\right) + \boldsymbol{r}_{a} z_{1}/\mathcal{R}_{a}, z_{1}\right]
 n\left[\boldsymbol{r}^{\prime}_{2}\left(1-z_{2}/\mathcal{R}_{b}\right) + \boldsymbol{r}_{b} z_{2}/\mathcal{R}_{b}, z_{2}\right] \right\rangle
\right\}
\end{aligned}
\end{equation}
The tilt covariance function is
\begin{equation}\label{tphase_n}
\begin{aligned}
&\left\langle \phi^{T}_{a}\left(\boldsymbol{r}_{1}\right) \phi^{T}_{b}\left(\boldsymbol{r}_{2}\right)\right\rangle = \\
&\quad\quad\quad\quad
\frac{16 k^{2}}{\pi^{2}R^{8}} \int_{0}^{\mathcal{R}_{a}}  \int_{0}^{\mathcal{R}_{b}} \iint dz_{1} \; dz_{2} \; d\boldsymbol{r}^{\prime}_{1}d\boldsymbol{r}^{\prime}_{2} W(\boldsymbol{r}^{\prime}_{1}/R) W(\boldsymbol{r}^{\prime}_{2}/R)
\left(\boldsymbol{r}_{1} \cdot \boldsymbol{r}^{\prime}_{1}\right)
\left(\boldsymbol{r}_{2} \cdot \boldsymbol{r}^{\prime}_{2}\right)
\\
&\quad\quad\quad\quad\quad\quad
\left\langle n\left[\boldsymbol{r}^{\prime}_{1}\left(1-z_{1}/\mathcal{R}_{a}\right) + \boldsymbol{r}_{a} z_{1}/\mathcal{R}_{a}, z_{1}\right]
 n\left[\boldsymbol{r}^{\prime}_{2}\left(1-z_{2}/\mathcal{R}_{b}\right) + \boldsymbol{r}_{b} z_{2}/\mathcal{R}_{b}, z_{2}\right] \right\rangle
\end{aligned}
\end{equation}
Finally, the covariance between the tilt and piston removed phase is given by 
\begin{equation}\label{tprphase_n}
\begin{aligned}
&\left\langle \phi_{a}\left(\boldsymbol{r}_{1}\right) \phi^{T}_{b}\left(\boldsymbol{r}_{2}\right)\right\rangle = \\
&\quad\quad\quad\quad
\frac{4 k^{2}}{\pi^{2}R^{6}} \int_{0}^{\mathcal{R}_{a}}  \int_{0}^{\mathcal{R}_{b}} \iint dz_{1} \; dz_{2} \; d\boldsymbol{r}^{\prime}_{1}d\boldsymbol{r}^{\prime}_{2} W(\boldsymbol{r}^{\prime}_{1}/R) W(\boldsymbol{r}^{\prime}_{2}/R)
\left(\boldsymbol{r}_{2} \cdot \boldsymbol{r}^{\prime}_{2}\right)
\\
&\quad\quad\quad\quad\quad\quad
\left\{
\left\langle n\left[\boldsymbol{r}_{1}\left(1-z_{1}/\mathcal{R}_{a}\right) + \boldsymbol{r}_{a} z_{1}/\mathcal{R}_{a}, z_{1}\right]
 n\left[\boldsymbol{r}^{\prime}_{2}\left(1-z_{2}/\mathcal{R}_{b}\right) + \boldsymbol{r}_{b} z_{2}/\mathcal{R}_{b}, z_{2}\right] \right\rangle - 
\right.
\\
&\quad\quad\quad\quad\quad\quad
\left.
\left\langle n\left[\boldsymbol{r}^{\prime}_{1}\left(1-z_{1}/\mathcal{R}_{a}\right) + \boldsymbol{r}_{a} z_{1}/\mathcal{R}_{a}, z_{1}\right]
 n\left[\boldsymbol{r}^{\prime}_{2}\left(1-z_{2}/\mathcal{R}_{b}\right) + \boldsymbol{r}_{b} z_{2}/\mathcal{R}_{b}, z_{2}\right] \right\rangle
\right\}
\end{aligned}
\end{equation}


A key result from Komolgorov theory is that the structure function for the index of refraction follows a two-thirds power law.
\begin{equation}\label{komolgorov}
\begin{aligned}
D_{n}\left(\boldsymbol{r}_{1}, z_{1}, \boldsymbol{r}_{2}, z_{2}\right) & = 
\left\langle \left\{n\left(\boldsymbol{r}_{1}, z_{1}\right) - n\left(\boldsymbol{r}_{2}, z_{2}\right)\right\}^{2} \right\rangle \\
& = 
C_{n}^{2}\left(\frac{z_{1} + z_{2}}{2}\right) \left[ \left\vert \boldsymbol{r}_{1} - \boldsymbol{r}_{2} \right\vert^{2} + \left\vert z_{1} - z_{2} \right\vert^{2}\right]^{1/3}
\end{aligned}
\end{equation}
Departures from the Komolgorov theory may be accommodated through use
of a different structure function.  For example, the effects of finite 
outer scale may be represented through the structure function 
\begin{equation}
\begin{aligned}
&D_{n}\left(\boldsymbol{r}_{1}, z_{1}, \boldsymbol{r}_{2}, z_{2}\right) = \\
&\quad\quad\quad C_{n}^{2}\left(\frac{z_{1} + z_{2}}{2}\right) \kappa_{0}^{-2/3}
\left\{\vphantom{\left(\left\vert \boldsymbol{r}_{1} - \boldsymbol{r}_{2} \right\vert^{2} + \left\vert z_{1} - z_{2} \right\vert\right)^{1/3} }
1.0468 - \right.\\
&\quad\quad\quad\quad 
\left.
0.62029\kappa_{0}^{1/3}\left(\left\vert \boldsymbol{r}_{1} - \boldsymbol{r}_{2} \right\vert^{2} + \left\vert z_{1} - z_{2} \right\vert\right)^{1/3} 
K_{1/3}\left(\kappa_{0}\left[\left\vert \boldsymbol{r}_{1} - \boldsymbol{r}_{2} \right\vert^{2} + \left\vert z_{1} - z_{2} \right\vert\right]\right)\right\}
\end{aligned}
\end{equation}
where $\kappa_{0} = 2\pi / L_{0}$, $L_{0}$ is the outer scale, and
$K_{1/3}\left(x\right)$ is a modified Bessel
function\cite{sasiela2007electromagnetic}.  A finite outer scale is
not considered further in this report.

The next step in this calculation is to use Equation \ref{komolgorov}
to express Equations \ref{prphase_n} - \ref{tprphase_n} in terms of
coordinate offsets rather than index of refraction covariance
functions.  We can rewrite Equation \ref{prphase_n} in a form that
will permit use of Equation \ref{komolgorov} via the expression
\begin{equation}\label{simplify}
\begin{split}
&\left\langle \left\{n\left[\boldsymbol{r}_{1},z_{1}\right] - n\left[\boldsymbol{r}^{\prime}_{1},z_{1}\right]\right\}n\left[\boldsymbol{r}_{2},z_{2}\right]\right\rangle -
\left\langle \left\{n\left[\boldsymbol{r}_{1},z_{1}\right] - n\left[\boldsymbol{r}^{\prime}_{1},z_{1}\right]\right\}n\left[\boldsymbol{r}^{\prime}_{2},z_{2}\right]\right\rangle = \\
&\quad\quad\quad\quad\quad\quad
-\frac{1}{2} \left(\left\langle \left\{n\left[\boldsymbol{r}_{1},z_{1}\right] - n\left[\boldsymbol{r}_{2},z_{2}\right]\right\}^{2}\right\rangle - 
\left\langle \left\{n\left[\boldsymbol{r}^{\prime}_{1},z_{1}\right] - n\left[\boldsymbol{r}_{2},z_{2}\right]\right\}^{2}\right\rangle\right) \\
&\quad\quad\quad\quad\quad\quad
+\frac{1}{2} \left(\left\langle \left\{n\left[\boldsymbol{r}_{1},z_{1}\right] - n\left[\boldsymbol{r}^{\prime}_{2},z_{2}\right]\right\}^{2}\right\rangle - 
\left\langle \left\{n\left[\boldsymbol{r}^{\prime}_{1},z_{1}\right] - n\left[\boldsymbol{r}^{\prime}_{2},z_{2}\right]\right\}^{2}\right\rangle\right) \\
\end{split}
\end{equation}
This substitution yields
\begin{equation}\label{prphase_n2}
\begin{aligned}
& \left\langle \vphantom{\hat{\phi}_{a}} \phi_{a}\left(\boldsymbol{r}_{1}\right) \phi_{b}\left(\boldsymbol{r}_{2}\right)\right\rangle = \\
&\quad\quad\quad\quad
-\frac{k^{2}}{2\pi^{2}R^{4}} \int_{0}^{\mathcal{R}_{a}}  \int_{0}^{\mathcal{R}_{b}} \iint dz_{1} \; dz_{2} \; d\boldsymbol{r}^{\prime}_{1}d\boldsymbol{r}^{\prime}_{2}  W(\boldsymbol{r}^{\prime}_{1}/R) W(\boldsymbol{r}^{\prime}_{2}/R)\\
&\quad\quad\quad\quad\quad\quad
\left(
\left\langle \left\{n\left[\boldsymbol{r}_{1}\left(1-z_{1}/\mathcal{R}_{a}\right) + \boldsymbol{r}_{a} z_{1}/\mathcal{R}_{a}, z_{1}\right] - 
 n\left[\boldsymbol{r}_{2}\left(1-z_{2}/\mathcal{R}_{b}\right) + \boldsymbol{r}_{b} z_{2}/\mathcal{R}_{b}, z_{2}\right] \right\}^{2} \right\rangle\right. - \\
&\quad\quad\quad\quad\quad\quad
\left\langle \left\{n\left[\boldsymbol{r}^{\prime}_{1}\left(1-z_{1}/\mathcal{R}_{a}\right) + \boldsymbol{r}_{a} z_{1}/\mathcal{R}_{a}, z_{1}\right] - 
 n\left[\boldsymbol{r}_{2}\left(1-z_{2}/\mathcal{R}_{b}\right) + \boldsymbol{r}_{b} z_{2}/\mathcal{R}_{b}, z_{2}\right] \right\}^{2} \right\rangle - \\
&\quad\quad\quad\quad\quad\quad
\left\langle \left\{n\left[\boldsymbol{r}_{1}\left(1-z_{1}/\mathcal{R}_{a}\right) + \boldsymbol{r}_{a} z_{1}/\mathcal{R}_{a}, z_{1}\right] - 
 n\left[\boldsymbol{r}^{\prime}_{2}\left(1-z_{2}/\mathcal{R}_{b}\right) + \boldsymbol{r}_{b} z_{2}/\mathcal{R}_{b}, z_{2}\right] \right\}^{2} \right\rangle + \\
&\quad\quad\quad\quad\quad\quad
\left.
\left\langle \left\{n\left[\boldsymbol{r}^{\prime}_{1}\left(1-z_{1}/\mathcal{R}_{a}\right) + \boldsymbol{r}_{a} z_{1}/\mathcal{R}_{a}, z_{1}\right] - 
 n\left[\boldsymbol{r}^{\prime}_{2}\left(1-z_{2}/\mathcal{R}_{b}\right) + \boldsymbol{r}_{b} z_{2}/\mathcal{R}_{b}, z_{2}\right] \right\}^{2} \right\rangle\right) \\
\end{aligned}
\end{equation}
Using Equation \ref{komolgorov}, this expression may be rewritten as
\begin{equation}\label{prphase_n3}
\begin{aligned}
& \left\langle \vphantom{\hat{\phi}_{a}} \phi_{a}\left(\boldsymbol{r}_{1}\right) \phi_{b}\left(\boldsymbol{r}_{2}\right)\right\rangle = \\
&\quad\quad\quad\quad
-\frac{k^{2}}{2\pi^{2}R^{4}} \int_{0}^{\mathcal{R}_{a}}  \int_{0}^{\mathcal{R}_{b}} \iint dz_{1} \; dz_{2} \; d\boldsymbol{r}^{\prime}_{1}d\boldsymbol{r}^{\prime}_{2}  
W(\boldsymbol{r}^{\prime}_{1}/R) W(\boldsymbol{r}^{\prime}_{2}/R) C_{n}^{2}\left(\frac{z_{1} + z_{2}}{2}\right) \\
&\quad\quad\quad\quad\quad\quad
\left(
\left[ \left\vert \boldsymbol{r}_{1}\left(1-z_{1}/\mathcal{R}_{a}\right) + \boldsymbol{r}_{a} z_{1}/\mathcal{R}_{a} - 
\boldsymbol{r}_{2}\left(1-z_{2}/\mathcal{R}_{b}\right) - \boldsymbol{r}_{b} z_{2}/\mathcal{R}_{b}\right\vert^{2} + 
\left\vert z_{1} - z_{2} \right\vert^{2}\right]^{1/3}
\right. -\\
& \quad\quad\quad\quad\quad\quad
\left[ \left\vert \boldsymbol{r}^{\prime}_{1}\left(1-z_{1}/\mathcal{R}_{a}\right) + \boldsymbol{r}_{a} z_{1}/\mathcal{R}_{a} - 
\boldsymbol{r}_{2}\left(1-z_{2}/\mathcal{R}_{b}\right) - \boldsymbol{r}_{b} z_{2}/\mathcal{R}_{b}\right\vert^{2} + 
\left\vert z_{1} - z_{2} \right\vert^{2}\right]^{1/3} - \\
& \quad\quad\quad\quad\quad\quad
\left[ \left\vert \boldsymbol{r}_{1}\left(1-z_{1}/\mathcal{R}_{a}\right) + \boldsymbol{r}_{a} z_{1}/\mathcal{R}_{a} - 
\boldsymbol{r}^{\prime}_{2}\left(1-z_{2}/\mathcal{R}_{b}\right) - \boldsymbol{r}_{b} z_{2}/\mathcal{R}_{b}\right\vert^{2} + 
\left\vert z_{1} - z_{2} \right\vert^{2}\right]^{1/3} + \\
& \quad\quad\quad\quad\quad\quad
\left.
\left[ \left\vert \boldsymbol{r}^{\prime}_{1}\left(1-z_{1}/\mathcal{R}_{a}\right) + \boldsymbol{r}_{a} z_{1}/\mathcal{R}_{a} - 
\boldsymbol{r}^{\prime}_{2}\left(1-z_{2}/\mathcal{R}_{b}\right) - \boldsymbol{r}_{b} z_{2}/\mathcal{R}_{b}\right\vert^{2} + 
\left\vert z_{1} - z_{2} \right\vert^{2}\right]^{1/3}\right)
\end{aligned}
\end{equation}

One may cast Equation \ref{tphase_n} into a similar form through the relationship
\begin{equation}
\begin{aligned}
&  \left\langle n\left[\boldsymbol{r}^{\prime}_{1}\left(1-z_{1}/\mathcal{R}_{a}\right) + \boldsymbol{r}_{a} z_{1}/\mathcal{R}_{a}, z_{1}\right]
n\left[\boldsymbol{r}^{\prime}_{2}\left(1-z_{2}/\mathcal{R}_{b}\right) + \boldsymbol{r}_{b} z_{2}/\mathcal{R}_{b}, z_{2}\right] \right\rangle = \\
& \quad\quad -\frac{1}{2}\left\langle \left\{ n\left[\boldsymbol{r}^{\prime}_{1}\left(1-z_{1}/\mathcal{R}_{a}\right) + \boldsymbol{r}_{a} z_{1}/\mathcal{R}_{a}, z_{1}\right] - 
 n\left[\boldsymbol{r}^{\prime}_{2}\left(1-z_{2}/\mathcal{R}_{b}\right) + \boldsymbol{r}_{b} z_{2}/\mathcal{R}_{b}, z_{2}\right]\right\}^{2} \right\rangle + \\
&\quad\quad\quad\quad \frac{1}{2}\left\langle \left\{ n\left[\boldsymbol{r}^{\prime}_{1}\left(1-z_{1}/\mathcal{R}_{a}\right)\right]\right\}^{2}\right\rangle + 
\frac{1}{2}\left\langle \left\{ n\left[\boldsymbol{r}^{\prime}_{2}\left(1-z_{2}/\mathcal{R}_{b}\right)\right]\right\}^{2}\right\rangle \\
\end{aligned}
\end{equation}
and by observing via asymmetry of the integrand that
\begin{equation}\label{intzero}
\begin{aligned}
\int d\boldsymbol{r}^{\prime}_{1} W\left(\boldsymbol{r}^{\prime}_{1}/R\right) \left(\boldsymbol{r}_{1}\cdot\boldsymbol{r}^{\prime}_{1}\right)
\left\langle \left\{ n\left[\boldsymbol{r}^{\prime}_{1}\left(1-z_{1}/\mathcal{R}_{a}\right)\right]\right\}^{2}\right\rangle
\int d\boldsymbol{r}^{\prime}_{2} W\left(\boldsymbol{r}^{\prime}_{2}/R\right) \left(\boldsymbol{r}_{2}\cdot\boldsymbol{r}^{\prime}_{2}\right) = 0 \\
\int d\boldsymbol{r}^{\prime}_{2} W\left(\boldsymbol{r}^{\prime}_{2}/R\right) \left(\boldsymbol{r}_{2}\cdot\boldsymbol{r}^{\prime}_{2}\right) 
\left\langle \left\{ n\left[\boldsymbol{r}^{\prime}_{2}\left(1-z_{2}/\mathcal{R}_{a}\right)\right]\right\}^{2}\right\rangle
\int d\boldsymbol{r}^{\prime}_{1} W\left(\boldsymbol{r}^{\prime}_{1}/R\right) \left(\boldsymbol{r}_{1}\cdot\boldsymbol{r}^{\prime}_{1}\right) = 0
\end{aligned}
\end{equation}
Consequently, Equation \ref{tphase_n} may be written 
\begin{equation}\label{tphase_n2}
\begin{aligned}
&\left\langle \phi^{T}_{a}\left(\boldsymbol{r}_{1}\right) \phi^{T}_{b}\left(\boldsymbol{r}_{2}\right)\right\rangle = \\
&\quad\quad\quad
-\frac{8k^{2}}{\pi^{2}R^{8}} \int_{0}^{\mathcal{R}_{a}}  \int_{0}^{\mathcal{R}_{b}} \iint dz_{1} \; dz_{2} \; d\boldsymbol{r}^{\prime}_{1}d\boldsymbol{r}^{\prime}_{2} W(\boldsymbol{r}^{\prime}_{1}/R) W(\boldsymbol{r}^{\prime}_{2}/R)
\left(\boldsymbol{r}_{1} \cdot \boldsymbol{r}^{\prime}_{1}\right)
\left(\boldsymbol{r}_{2} \cdot \boldsymbol{r}^{\prime}_{2}\right)
\\
&\quad\quad\quad
\left\langle \left\{ n\left[\boldsymbol{r}^{\prime}_{1}\left(1-z_{1}/\mathcal{R}_{a}\right) + \boldsymbol{r}_{a} z_{1}/\mathcal{R}_{a}, z_{1}\right] - 
 n\left[\boldsymbol{r}^{\prime}_{2}\left(1-z_{2}/\mathcal{R}_{b}\right) + \boldsymbol{r}_{b} z_{2}/\mathcal{R}_{b}, z_{2}\right]\right\}^{2} \right\rangle \\
\end{aligned}
\end{equation}
Using Equation \ref{komolgorov}, this becomes
\begin{equation}\label{tphase_n3}
\begin{aligned}
&\left\langle \phi^{T}_{a}\left(\boldsymbol{r}_{1}\right) \phi^{T}_{b}\left(\boldsymbol{r}_{2}\right)\right\rangle = \\
&\quad\quad\quad
-\frac{8k^{2}}{\pi^{2}R^{8}} \int_{0}^{\mathcal{R}_{a}}  \int_{0}^{\mathcal{R}_{b}} \iint dz_{1} \; dz_{2} \; d\boldsymbol{r}^{\prime}_{1}d\boldsymbol{r}^{\prime}_{2} W(\boldsymbol{r}^{\prime}_{1}/R) W(\boldsymbol{r}^{\prime}_{2}/R)
\left(\boldsymbol{r}_{1} \cdot \boldsymbol{r}^{\prime}_{1}\right)
\left(\boldsymbol{r}_{2} \cdot \boldsymbol{r}^{\prime}_{2}\right)
C_{n}^{2}\left(\frac{z_{1} + z_{2}}{2}\right) 
\\
&\quad\quad\quad
\left[ \left\vert \boldsymbol{r}^{\prime}_{1}\left(1-z_{1}/\mathcal{R}_{a}\right) + \boldsymbol{r}_{a} z_{1}/\mathcal{R}_{a} - 
 \boldsymbol{r}^{\prime}_{2}\left(1-z_{2}/\mathcal{R}_{b}\right) - \boldsymbol{r}_{b} z_{2}/\mathcal{R}_{b}\right\vert^{2} + 
\left\vert z_{1} - z_{2} \right\vert^{2} \right]^{1/3}
\end{aligned}
\end{equation}

Finally, the covariances between the tilt and piston removed phase in Equation \ref{tprphase_n} may be rewritten as 
\begin{equation}\label{tprphase_n2}
\begin{aligned}
&\left\langle \phi_{a}\left(\boldsymbol{r}_{1}\right) \phi^{T}_{b}\left(\boldsymbol{r}_{2}\right)\right\rangle = \\
&\quad\quad\quad\quad
-\frac{2k^{2}}{\pi^{2}R^{6}} \int_{0}^{\mathcal{R}_{a}}  \int_{0}^{\mathcal{R}_{b}} \iint dz_{1} \; dz_{2} \; d\boldsymbol{r}^{\prime}_{1}d\boldsymbol{r}^{\prime}_{2} W(\boldsymbol{r}^{\prime}_{1}/R) W(\boldsymbol{r}^{\prime}_{2}/R)
\left(\boldsymbol{r}_{2} \cdot \boldsymbol{r}^{\prime}_{2}\right)
\\
&\quad\quad\quad\quad\quad\quad
\left\{
\left\langle \left\{n\left[\boldsymbol{r}_{1}\left(1-z_{1}/\mathcal{R}_{a}\right) + \boldsymbol{r}_{a} z_{1}/\mathcal{R}_{a}, z_{1}\right] -
 n\left[\boldsymbol{r}^{\prime}_{2}\left(1-z_{2}/\mathcal{R}_{b}\right) + \boldsymbol{r}_{b} z_{2}/\mathcal{R}_{b}, z_{2}\right] \right\}^{2}\right\rangle 
\right. -
\\
&\quad\quad\quad\quad\quad\quad
\left.
\left\langle \left\{ n\left[\boldsymbol{r}^{\prime}_{1}\left(1-z_{1}/\mathcal{R}_{a}\right) + \boldsymbol{r}_{a} z_{1}/\mathcal{R}_{a}, z_{1}\right]- 
 n\left[\boldsymbol{r}^{\prime}_{2}\left(1-z_{2}/\mathcal{R}_{b}\right) + \boldsymbol{r}_{b} z_{2}/\mathcal{R}_{b}, z_{2}\right] \right\}^{2}\right\rangle
\right\}
\end{aligned}
\end{equation}
Again using Equation \ref{komolgorov}, this may be expressed as
\begin{equation}\label{tprphase_n3}
\begin{aligned}
&\left\langle \phi_{a}\left(\boldsymbol{r}_{1}\right) \phi^{T}_{b}\left(\boldsymbol{r}_{2}\right)\right\rangle = \\
&\quad\quad\quad\quad
-\frac{2k^{2}}{\pi^{2}R^{6}} \int_{0}^{\mathcal{R}_{a}}  \int_{0}^{\mathcal{R}_{b}} \iint dz_{1} \; dz_{2} \; d\boldsymbol{r}^{\prime}_{1}d\boldsymbol{r}^{\prime}_{2} W(\boldsymbol{r}^{\prime}_{1}/R) W(\boldsymbol{r}^{\prime}_{2}/R)
\left(\boldsymbol{r}^{\prime}_{2} \cdot \boldsymbol{r}^{\prime}_{2}\right)
C_{n}^{2}\left(\frac{z_{1} + z_{2}}{2}\right) 
\\
&\quad\quad\quad\quad\quad\quad
\left(
\left[\left\vert \boldsymbol{r}_{1}\left(1-z_{1}/\mathcal{R}_{a}\right) + \boldsymbol{r}_{a} z_{1}/\mathcal{R}_{a} - \boldsymbol{r}^{\prime}_{2}\left(1-z_{2}/\mathcal{R}_{b}\right) - \boldsymbol{r}_{b} z_{2}/\mathcal{R}_{b}, z_{2}\right\vert^{2}   + 
\left\vert z_{1} - z_{2} \right\vert^{2}\right]^{1/3}
\right. -
\\
&\quad\quad\quad\quad\quad\quad
\left.
\left[ \left\vert \boldsymbol{r}^{\prime}_{1}\left(1-z_{1}/\mathcal{R}_{a}\right) + \boldsymbol{r}_{a} z_{1}/\mathcal{R}_{a} - 
 \boldsymbol{r}^{\prime}_{2}\left(1-z_{2}/\mathcal{R}_{b}\right) - \boldsymbol{r}_{b} z_{2}/\mathcal{R}_{b}\right\vert^{2} + 
\left\vert z_{1} - z_{2} \right\vert^{2}\right]^{1/3}
\right)
\end{aligned}
\end{equation}

Equations \ref{prphase_n3}, \ref{tphase_n3} and \ref{tprphase_n3}
constitute predictions for the two point covariance functions derived
from the theory of Komolgorov turbulence.  These functions may be
numerically calculated for arbitrary source geometries, telescope
aperture diameters, and $C_{n}^{2}(z)$ profiles.  However, further
simplification of these expressions is possible.  To proceed, consider
the coordinate transformation
\begin{equation}
\begin{aligned}
z = \frac{z_{1}+z_{2}}{2} \quad & \quad z_{-} = z_{1}-z_{2} \\
z_{1} = z+\frac{z_{-}}{2} \quad & \quad z_{2} = z-\frac{z_{-}}{2} \\
\end{aligned}
\end{equation}
Tyler invokes the assumption that only contributions to the integral
arise when $z_{-} \approx 0$ due to the rapid falloff in the
covariance function, and makes the approximations
\begin{equation}
\begin{aligned}
\frac{z_{1}}{\mathcal{R}_{a}} \approx \frac{z}{\mathcal{R}_{a}} \quad & \quad \frac{z_{2}}{\mathcal{R}_{b}} \approx \frac{z}{\mathcal{R}_{b}} 
\end{aligned}
\end{equation}
Sasiela\cite{sasiela2007electromagnetic} provides a complementary and
more detailed description in Section 2.3.  With this approximation,
the limits of integration on the variable $z$ extend to the minimum
range $\mathcal{R}_{\rm min} = {\rm min}\left[\mathcal{R}_{a},
  \mathcal{R}_{b}\right]$.  The limits of integration on the variable
$z_{-}$ may be extended to $\pm\infty$ without affecting the value of
the integral.  Ergo, Equations \ref{prphase_n3},
\ref{tphase_n3} and \ref{tprphase_n3} may be written as
\begin{equation}\label{prphase_n4}
\begin{aligned}
& \left\langle \vphantom{\hat{\phi}_{a}} \phi_{a}\left(\boldsymbol{r}_{1}\right) \phi_{b}\left(\boldsymbol{r}_{2}\right)\right\rangle = \\
&\quad\quad\quad
-\frac{k^{2}}{2\pi^{2}R^{4}} \int_{0}^{\mathcal{R}_{\rm min}} dz\;C_{n}^{2}\left(z\right)
\int_{-\infty}^{\infty} dz_{-}\\
&\quad\quad\quad\quad\quad 
\iint \; d\boldsymbol{r}^{\prime}_{1}d\boldsymbol{r}^{\prime}_{2}  
W(\boldsymbol{r}^{\prime}_{1}/R) W(\boldsymbol{r}^{\prime}_{2}/R) \\
&\quad\quad\quad\quad\quad\quad
\left(
\left[ \left\vert \boldsymbol{r}_{1}\left(1-z/\mathcal{R}_{a}\right) + \boldsymbol{r}_{a} z/\mathcal{R}_{a} - 
\boldsymbol{r}_{2}\left(1-z/\mathcal{R}_{b}\right) - \boldsymbol{r}_{b} z/\mathcal{R}_{b}\right\vert^{2} + 
\left\vert z_{-} \right\vert^{2}\right]^{1/3} -
\right. \\
& \quad\quad\quad\quad\quad\quad
\left[ \left\vert \boldsymbol{r}^{\prime}_{1}\left(1-z/\mathcal{R}_{a}\right) + \boldsymbol{r}_{a} z/\mathcal{R}_{a} - 
\boldsymbol{r}_{2}\left(1-z/\mathcal{R}_{b}\right) - \boldsymbol{r}_{b} z/\mathcal{R}_{b}\right\vert^{2} + 
\left\vert z_{-} \right\vert^{2}\right]^{1/3} - \\
& \quad\quad\quad\quad\quad\quad
\left[ \left\vert \boldsymbol{r}_{1}\left(1-z/\mathcal{R}_{a}\right) + \boldsymbol{r}_{a} z/\mathcal{R}_{a} - 
\boldsymbol{r}^{\prime}_{2}\left(1-z/\mathcal{R}_{b}\right) - \boldsymbol{r}_{b} z/\mathcal{R}_{b}\right\vert^{2} + 
\left\vert z_{-} \right\vert^{2}\right]^{1/3} + \\
& \quad\quad\quad\quad\quad\quad
\left.
\left[ \left\vert \boldsymbol{r}^{\prime}_{1}\left(1-z/\mathcal{R}_{a}\right) + \boldsymbol{r}_{a} z/\mathcal{R}_{a} - 
\boldsymbol{r}^{\prime}_{2}\left(1-z/\mathcal{R}_{b}\right) - \boldsymbol{r}_{b} z/\mathcal{R}_{b}\right\vert^{2} + 
\left\vert z_{-} \right\vert^{2}\right]^{1/3}\right)
\end{aligned}
\end{equation}

\begin{equation}\label{tphase_n4}
\begin{aligned}
&\left\langle \phi^{T}_{a}\left(\boldsymbol{r}_{1}\right) \phi^{T}_{b}\left(\boldsymbol{r}_{2}\right)\right\rangle = \\
&\quad\quad\quad
-\frac{8k^{2}}{\pi^{2}R^{8}} \int_{0}^{\mathcal{R}_{\rm min}} dz\;C_{n}^{2}\left(z\right)
\int_{-\infty}^{\infty} dz_{-} \\
&\quad\quad\quad\quad\quad 
\iint \; d\boldsymbol{r}^{\prime}_{1}d\boldsymbol{r}^{\prime}_{2} W(\boldsymbol{r}^{\prime}_{1}/R) W(\boldsymbol{r}^{\prime}_{2}/R)
\left(\boldsymbol{r}_{1} \cdot \boldsymbol{r}^{\prime}_{1}\right)
\left(\boldsymbol{r}_{2} \cdot \boldsymbol{r}^{\prime}_{2}\right)
\\
&\quad\quad\quad\quad\quad\quad
\left[ \left\vert \boldsymbol{r}^{\prime}_{1}\left(1-z/\mathcal{R}_{a}\right) + \boldsymbol{r}_{a} z/\mathcal{R}_{a} - 
 \boldsymbol{r}^{\prime}_{2}\left(1-z/\mathcal{R}_{b}\right) - \boldsymbol{r}_{b} z/\mathcal{R}_{b}\right\vert^{2} + 
\left\vert z_{-} \right\vert^{2} \right]^{1/3}
\end{aligned}
\end{equation}

\begin{equation}\label{tprphase_n4}
\begin{aligned}
&\left\langle \phi_{a}\left(\boldsymbol{r}_{1}\right) \phi^{T}_{b}\left(\boldsymbol{r}_{2}\right)\right\rangle = \\
&\quad\quad\quad
-\frac{2k^{2}}{\pi^{2}R^{6}} \int_{0}^{\mathcal{R}_{\rm min}} dz\;C_{n}^{2}\left(z\right) 
\int_{-\infty}^{\infty} dz_{-} \\
&\quad\quad\quad\quad\quad 
\iint  \; d\boldsymbol{r}^{\prime}_{1}d\boldsymbol{r}^{\prime}_{2} W(\boldsymbol{r}^{\prime}_{1}/R) W(\boldsymbol{r}^{\prime}_{2}/R)
\left(\boldsymbol{r}^{\prime}_{2} \cdot \boldsymbol{r}^{\prime}_{2}\right)
\\
&\quad\quad\quad\quad\quad\quad
\left(
\left[\left\vert \boldsymbol{r}_{1}\left(1-z/\mathcal{R}_{a}\right) + \boldsymbol{r}_{a} z/\mathcal{R}_{a} - \boldsymbol{r}^{\prime}_{2}\left(1-z/\mathcal{R}_{b}\right) - \boldsymbol{r}_{b} z/\mathcal{R}_{b}\right\vert^{2}   + 
\left\vert z_{-} \right\vert^{2}\right]^{1/3}
\right. -
\\
&\quad\quad\quad\quad\quad\quad
\left.
\left[ \left\vert \boldsymbol{r}^{\prime}_{1}\left(1-z/\mathcal{R}_{a}\right) + \boldsymbol{r}_{a} z/\mathcal{R}_{a} - 
 \boldsymbol{r}^{\prime}_{2}\left(1-z/\mathcal{R}_{b}\right) - \boldsymbol{r}_{b} z/\mathcal{R}_{b}\right\vert^{2} + 
\left\vert z_{-} \right\vert^{2}\right]^{1/3}
\right)
\end{aligned}
\end{equation}

The next step in this calculation is to integrate Equations
\ref{prphase_n4} - \ref{tprphase_n4} over the variable $z_{-}$
using the expression 
\begin{equation}\label{intglident}
\int_{-\infty}^{\infty} dz\left[\left(A^{2} + z^{2}\right)^{1/3} - z^{2/3}\right] = 
\frac{2^{1/3}}{5}\frac{\left[\Gamma\left(\frac{1}{6}\right)\right]^{2}}{\Gamma\left(\frac{1}{3}\right)} A^{5/3}
\end{equation}
The quantity $\left\vert z_{-}\right\vert^{2/3}$ may be added and
subtracted pairwise to Equations \ref{prphase_n4} and
\ref{tprphase_n4} so as to produce integrands of the form shown in
Equation \ref{intglident}.  Appealing to Equation \ref{intzero}, this
same quantity may be added to Equation \ref{tphase_n4} without
changing the value of the integrand.  This permits integration over
$z_{-}$ via the relationship The expressions for the piston removed
phase covariance, tilt covariance, and cross covariance become
\begin{equation}\label{prphase_n5}
\begin{aligned}
&\left\langle \vphantom{\hat{\phi}_{a}} \phi_{a}\left(\boldsymbol{r}_{1}\right) \phi_{b}\left(\boldsymbol{r}_{2}\right)\right\rangle = \\
&\quad\quad\quad
-\frac{k^{2}}{2\pi^{2}R^{4}} 
\frac{2^{1/3}}{5}\frac{\left[\Gamma\left(\frac{1}{6}\right)\right]^{2}}{\Gamma\left(\frac{1}{3}\right)} 
\int_{0}^{\mathcal{R}_{\rm min}} dz\;C_{n}^{2}\left(z\right) \\
&\quad\quad\quad\quad\quad
\iint \; d\boldsymbol{r}^{\prime}_{1}d\boldsymbol{r}^{\prime}_{2}  
W(\boldsymbol{r}^{\prime}_{1}/R) W(\boldsymbol{r}^{\prime}_{2}/R) \\
&\quad\quad\quad\quad\quad\quad
\left(
\left\vert \boldsymbol{r}_{1}\left(1-z/\mathcal{R}_{a}\right) + \boldsymbol{r}_{a} z/\mathcal{R}_{a} - 
\boldsymbol{r}_{2}\left(1-z/\mathcal{R}_{b}\right) - \boldsymbol{r}_{b} z/\mathcal{R}_{b}\right\vert^{5/3} -
\right. \\
& \quad\quad\quad\quad\quad\quad
\left\vert \boldsymbol{r}^{\prime}_{1}\left(1-z/\mathcal{R}_{a}\right) + \boldsymbol{r}_{a} z/\mathcal{R}_{a} - 
\boldsymbol{r}_{2}\left(1-z/\mathcal{R}_{b}\right) - \boldsymbol{r}_{b} z/\mathcal{R}_{b}\right\vert^{5/3} - \\
& \quad\quad\quad\quad\quad\quad
\left\vert \boldsymbol{r}_{1}\left(1-z/\mathcal{R}_{a}\right) + \boldsymbol{r}_{a} z/\mathcal{R}_{a} - 
\boldsymbol{r}^{\prime}_{2}\left(1-z/\mathcal{R}_{b}\right) - \boldsymbol{r}_{b} z/\mathcal{R}_{b}\right\vert^{5/3} + \\
& \quad\quad\quad\quad\quad\quad
\left.
\left\vert \boldsymbol{r}^{\prime}_{1}\left(1-z/\mathcal{R}_{a}\right) + \boldsymbol{r}_{a} z/\mathcal{R}_{a} - 
\boldsymbol{r}^{\prime}_{2}\left(1-z/\mathcal{R}_{b}\right) - \boldsymbol{r}_{b} z/\mathcal{R}_{b}\right\vert^{5/3}\right)
\end{aligned}
\end{equation}

\begin{equation}\label{tphase_n5}
\begin{aligned}
&\left\langle \phi^{T}_{a}\left(\boldsymbol{r}_{1}\right) \phi^{T}_{b}\left(\boldsymbol{r}_{2}\right)\right\rangle = \\
&\quad\quad\quad
-\frac{8k^{2}}{\pi^{2}R^{8}} 
\frac{2^{1/3}}{5}\frac{\left[\Gamma\left(\frac{1}{6}\right)\right]^{2}}{\Gamma\left(\frac{1}{3}\right)} 
\int_{0}^{\mathcal{R}_{\rm min}} dz\;C_{n}^{2}\left(z\right) \\
&\quad\quad\quad\quad\quad
\iint \; d\boldsymbol{r}^{\prime}_{1}d\boldsymbol{r}^{\prime}_{2} W(\boldsymbol{r}^{\prime}_{1}/R) W(\boldsymbol{r}^{\prime}_{2}/R)
\left(\boldsymbol{r}_{1} \cdot \boldsymbol{r}^{\prime}_{1}\right)
\left(\boldsymbol{r}_{2} \cdot \boldsymbol{r}^{\prime}_{2}\right)
\\
&\quad\quad\quad\quad\quad\quad
\left\vert \boldsymbol{r}^{\prime}_{1}\left(1-z/\mathcal{R}_{a}\right) + \boldsymbol{r}_{a} z/\mathcal{R}_{a} - 
 \boldsymbol{r}^{\prime}_{2}\left(1-z/\mathcal{R}_{b}\right) - \boldsymbol{r}_{b} z/\mathcal{R}_{b}\right\vert^{5/3}
\end{aligned}
\end{equation}

\begin{equation}\label{tprphase_n5}
\begin{aligned}
&\left\langle \phi_{a}\left(\boldsymbol{r}_{1}\right) \phi^{T}_{b}\left(\boldsymbol{r}_{2}\right)\right\rangle = \\
&\quad\quad\quad
-\frac{2k^{2}}{\pi^{2}R^{6}} 
\frac{2^{1/3}}{5}\frac{\left[\Gamma\left(\frac{1}{6}\right)\right]^{2}}{\Gamma\left(\frac{1}{3}\right)} 
\int_{0}^{\mathcal{R}_{\rm min}} dz\;C_{n}^{2}\left(z\right) \\
&\quad\quad\quad\quad\quad
\iint  \; d\boldsymbol{r}^{\prime}_{1}d\boldsymbol{r}^{\prime}_{2} W(\boldsymbol{r}^{\prime}_{1}/R) W(\boldsymbol{r}^{\prime}_{2}/R)
\left(\boldsymbol{r}^{\prime}_{2} \cdot \boldsymbol{r}^{\prime}_{2}\right)
\\
&\quad\quad\quad\quad\quad\quad
\left(
\left\vert \boldsymbol{r}_{1}\left(1-z/\mathcal{R}_{a}\right) + \boldsymbol{r}_{a} z/\mathcal{R}_{a} - \boldsymbol{r}^{\prime}_{2}\left(1-z/\mathcal{R}_{b}\right) - \boldsymbol{r}_{b} z/\mathcal{R}_{b}\right\vert^{5/3}
\right. -
\\
&\quad\quad\quad\quad\quad\quad
\left.
\left\vert \boldsymbol{r}^{\prime}_{1}\left(1-z/\mathcal{R}_{a}\right) + \boldsymbol{r}_{a} z/\mathcal{R}_{a} - 
 \boldsymbol{r}^{\prime}_{2}\left(1-z/\mathcal{R}_{b}\right) - \boldsymbol{r}_{b} z/\mathcal{R}_{b}\right\vert^{5/3}
\right)
\end{aligned}
\end{equation}


Define the following two variables
\begin{equation}\label{Omega}
\boldsymbol{\Omega}_{ab}\left(z\right) = 
\left(\frac{\boldsymbol{r}_{a}}{\mathcal{R}_{a}} - \frac{\boldsymbol{r}_{b}}{\mathcal{R}_{b}} \right) \frac{z/R}{1-z/\mathcal{R}_{\rm a}}
\end{equation}
\begin{equation}\label{Q}
Q_{ab}\left(z\right) =  
\frac{1 - z/\mathcal{R}_{\rm b}}{1-z/\mathcal{R}_{\rm a}}
\end{equation}
to arrive at
\begin{equation}\label{prphase_n6}
\begin{aligned}
&\left\langle \vphantom{\hat{\phi}_{a}} \phi_{a}\left(\boldsymbol{r}_{1}\right) \phi_{b}\left(\boldsymbol{r}_{2}\right)\right\rangle = \\
&\quad\quad\quad
-\frac{k^{2}}{2\pi^{2}R^{4}} 
\frac{2^{1/3}}{5}\frac{\left[\Gamma\left(\frac{1}{6}\right)\right]^{2}}{\Gamma\left(\frac{1}{3}\right)} 
\int_{0}^{\mathcal{R}_{\rm min}} dz\;C_{n}^{2}\left(z\right) 
\left(1 - z/\mathcal{R}_{\rm a}\right)^{5/3}\\
&\quad\quad\quad\quad\quad
\iint \; d\boldsymbol{r}^{\prime}_{1}d\boldsymbol{r}^{\prime}_{2}  
W(\boldsymbol{r}^{\prime}_{1}/R) W(\boldsymbol{r}^{\prime}_{2}/R) \\
&\quad\quad\quad\quad\quad\quad
\left(\left\vert \boldsymbol{r}_{1}  - Q_{ab}\left(z\right)\boldsymbol{r}_{2} - \boldsymbol{\Omega}_{ab}\left(z\right)R\right\vert^{5/3} -
\left\vert \boldsymbol{r}^{\prime}_{1}  - Q_{ab}\left(z\right)\boldsymbol{r}_{2} - \boldsymbol{\Omega}_{ab}\left(z\right)R\right\vert^{5/3} -
\right. \\
& \quad\quad\quad\quad\quad\quad
\left.
\left\vert \boldsymbol{r}_{1}  - Q_{ab}\left(z\right)\boldsymbol{r}^{\prime}_{2} - \boldsymbol{\Omega}_{ab}\left(z\right)R\right\vert^{5/3} + 
\left\vert \boldsymbol{r}^{\prime}_{1}  - Q_{ab}\left(z\right)\boldsymbol{r}^{\prime}_{2} - \boldsymbol{\Omega}_{ab}\left(z\right)R\right\vert^{5/3}\right)
\end{aligned}
\end{equation}

\begin{equation}\label{tphase_n6}
\begin{aligned}
&\left\langle \phi^{T}_{a}\left(\boldsymbol{r}_{1}\right) \phi^{T}_{b}\left(\boldsymbol{r}_{2}\right)\right\rangle = \\
&\quad\quad\quad
-\frac{8k^{2}}{\pi^{2}R^{8}} 
\frac{2^{1/3}}{5}\frac{\left[\Gamma\left(\frac{1}{6}\right)\right]^{2}}{\Gamma\left(\frac{1}{3}\right)} 
\int_{0}^{\mathcal{R}_{\rm min}} dz\;C_{n}^{2}\left(z\right) 
\left(1 - z/\mathcal{R}_{\rm a}\right)^{5/3}\\
&\quad\quad\quad\quad\quad
\iint \; d\boldsymbol{r}^{\prime}_{1}d\boldsymbol{r}^{\prime}_{2} W(\boldsymbol{r}^{\prime}_{1}/R) W(\boldsymbol{r}^{\prime}_{2}/R)
\left(\boldsymbol{r}_{1} \cdot \boldsymbol{r}^{\prime}_{1}\right)
\left(\boldsymbol{r}_{2} \cdot \boldsymbol{r}^{\prime}_{2}\right)
\\
&\quad\quad\quad\quad\quad\quad
\left\vert \boldsymbol{r}^{\prime}_{1} - Q_{ab}\left(z\right) \boldsymbol{r}^{\prime}_{2} + \boldsymbol{\Omega}_{ab}\left(z\right) R\right\vert^{5/3}
\end{aligned}
\end{equation}

\begin{equation}\label{tprphase_n6}
\begin{aligned}
&\left\langle \phi_{a}\left(\boldsymbol{r}_{1}\right) \phi^{T}_{b}\left(\boldsymbol{r}_{2}\right)\right\rangle = \\
&\quad\quad\quad
-\frac{2k^{2}}{\pi^{2}R^{6}} 
\frac{2^{1/3}}{5}\frac{\left[\Gamma\left(\frac{1}{6}\right)\right]^{2}}{\Gamma\left(\frac{1}{3}\right)} 
\int_{0}^{\mathcal{R}_{\rm min}} dz\;C_{n}^{2}\left(z\right) 
\left(1 - z/\mathcal{R}_{\rm a}\right)^{5/3}\\
&\quad\quad\quad\quad\quad
\iint  \; d\boldsymbol{r}^{\prime}_{1}d\boldsymbol{r}^{\prime}_{2} W(\boldsymbol{r}^{\prime}_{1}/R) W(\boldsymbol{r}^{\prime}_{2}/R)
\left(\boldsymbol{r}_{2} \cdot \boldsymbol{r}^{\prime}_{2}\right)
\\
&\quad\quad\quad\quad\quad\quad
\left(
\left\vert \boldsymbol{r}_{1} - Q_{ab}\left(z\right)\boldsymbol{r}^{\prime}_{2} + \boldsymbol{\Omega}_{ab}\left(z\right) R\right\vert^{5/3}
- \left\vert \boldsymbol{r}^{\prime}_{1}  - 
 Q_{ab}\left(z\right)\boldsymbol{r}^{\prime}_{2} + \boldsymbol{\Omega}_{ab}\left(z\right) R \right\vert^{5/3}
\right)
\end{aligned}
\end{equation}

As a final step, make the change of variables
\begin{equation}
\begin{aligned}
\boldsymbol{\rho}_{1} = \boldsymbol{r}_{1}/R \quad & \quad \boldsymbol{\rho}_{2} = \boldsymbol{r}_{2}/R \\
\boldsymbol{\rho}^{\prime}_{1} = \boldsymbol{r}^{\prime}_{1}/R \quad & \quad \boldsymbol{\rho}^{\prime}_{2} = \boldsymbol{r}^{\prime}_{2}/R
\end{aligned}
\end{equation}
and define the quantity 
\begin{equation}\label{Xi}
\Xi = \frac{1}{5}\left(\frac{1}{2}\right)^{7/3} \frac{\left[\Gamma\left(\frac{1}{6}\right)\right]^{2}}{\Gamma\left(\frac{1}{3}\right)} = 0.458986
\end{equation}
to obtain
\begin{equation}\label{prphase_n7}
\begin{aligned}
&\left\langle \vphantom{\hat{\phi}_{a}} \phi_{a}\left(R\boldsymbol{\rho}_{1}\right) \phi_{b}\left(R\boldsymbol{\rho}_{2}\right)\right\rangle = \\ 
&\quad\quad\quad
 -\frac{k^{2}}{\pi^{2}} \Xi D^{5/3}
\int_{0}^{\mathcal{R}_{\rm min}} dz\;C_{n}^{2}\left(z\right) \left(1 - z/\mathcal{R}_{\rm a}\right)^{5/3}\\
& \quad\quad\quad\quad\quad
\iint \; d\boldsymbol{\rho}^{\prime}_{1}d\boldsymbol{\rho}^{\prime}_{2}  
W(\boldsymbol{\rho}^{\prime}_{1}) W(\boldsymbol{\rho}^{\prime}_{2}) \\
& \quad\quad\quad\quad\quad\quad
\left(\left\vert \boldsymbol{\rho}_{1}  - Q_{ab}\left(z\right)\boldsymbol{\rho}_{2} - \boldsymbol{\Omega}_{ab}\left(z\right)\right\vert^{5/3} -
\left\vert \boldsymbol{\rho}^{\prime}_{1}  - Q_{ab}\left(z\right)\boldsymbol{\rho}_{2} - \boldsymbol{\Omega}_{ab}\left(z\right)\right\vert^{5/3} -
\right. \\
& \quad\quad\quad\quad\quad\quad
\left.
\left\vert \boldsymbol{\rho}_{1}  - Q_{ab}\left(z\right)\boldsymbol{\rho}^{\prime}_{2} - \boldsymbol{\Omega}_{ab}\left(z\right)\right\vert^{5/3} + 
\left\vert \boldsymbol{\rho}^{\prime}_{1}  - Q_{ab}\left(z\right)\boldsymbol{\rho}^{\prime}_{2} - \boldsymbol{\Omega}_{ab}\left(z\right)\right\vert^{5/3}\right)
\end{aligned}
\end{equation}

\begin{equation}\label{tphase_n7}
\begin{aligned}
&\left\langle \phi^{T}_{a}\left(R\boldsymbol{\rho}_{1}\right) \phi^{T}_{b}\left(R\boldsymbol{\rho}_{2}\right)\right\rangle = \\
&\quad\quad\quad
-\frac{16k^{2}}{\pi^{2}} 
\Xi D^{5/3}
\int_{0}^{\mathcal{R}_{\rm min}} dz\;C_{n}^{2}\left(z\right) \left(1 - z/\mathcal{R}_{\rm a}\right)^{5/3}\\
& \quad\quad\quad\quad\quad
\iint \; d\boldsymbol{\rho}^{\prime}_{1}d\boldsymbol{\rho}^{\prime}_{2} W(\boldsymbol{\rho}^{\prime}_{1}) W(\boldsymbol{\rho}^{\prime}_{2})
\left(\boldsymbol{\rho}_{1} \cdot \boldsymbol{\rho}^{\prime}_{1}\right)
\left(\boldsymbol{\rho}_{2} \cdot \boldsymbol{\rho}^{\prime}_{2}\right)
\\
&\quad\quad\quad\quad\quad\quad
\left\vert \boldsymbol{\rho}^{\prime}_{1} - Q_{ab}\left(z\right) \boldsymbol{\rho}^{\prime}_{2} + \boldsymbol{\Omega}_{ab}\left(z\right)\right\vert^{5/3}
\end{aligned}
\end{equation}

\begin{equation}\label{tprphase_n7}
\begin{aligned}
&\left\langle \phi_{a}\left(R\boldsymbol{\rho}_{1}\right) \phi^{T}_{b}\left(R\boldsymbol{\rho}_{2}\right)\right\rangle = \\
&\quad\quad\quad
-\frac{4k^{2}}{\pi^{2}} 
\Xi D^{5/3}
\int_{0}^{\mathcal{R}_{\rm min}} dz\;C_{n}^{2}\left(z\right) \left(1 - z/\mathcal{R}_{\rm a}\right)^{5/3}\\
& \quad\quad\quad\quad\quad
\iint  \; d\boldsymbol{\rho}^{\prime}_{1}d\boldsymbol{\rho}^{\prime}_{2} W(\boldsymbol{\rho}^{\prime}_{1}) W(\boldsymbol{\rho}^{\prime}_{2})
\left(\boldsymbol{\rho}_{2} \cdot \boldsymbol{\rho}^{\prime}_{2}\right)
C_{n}^{2}\left(z\right) 
\\
&\quad\quad\quad\quad\quad\quad
\left(
\left\vert \boldsymbol{\rho}_{1} - Q_{ab}\left(z\right)\boldsymbol{\rho}^{\prime}_{2} + \boldsymbol{\Omega}_{ab}\left(z\right) \right\vert^{5/3}
- \left\vert \boldsymbol{\rho}^{\prime}_{1}  -  Q_{ab}\left(z\right)\boldsymbol{\rho}^{\prime}_{2} + \boldsymbol{\Omega}_{ab}\left(z\right)  \right\vert^{5/3}
\right)
\end{aligned}
\end{equation}

Equation \ref{prphase_n7} represents the two point piston removed
phase covariance function for two sources at arbitrary ranges and
locations within the field.  The parameters that enter into the
calculation of this quantity are the wavenumber $k$, aperture diameter
$D$, turbulence profile $C_{n}^{2}(z)$, source ranges
$\mathcal{R}_{a}$ and $\mathcal{R}_{b}$, and their locations
$\boldsymbol{r}_{a}$ and $\boldsymbol{r}_{b}$ in the field.  All of
these parameters are known in a specific observing geometry such as
that shown in Figure \ref{geo} with the exception of the turbulence
profile $C_{n}^{2}(z)$.  A measurement of $C_{n}^{2}(z)$ is available
from the MASS/DIMM.  Alternatively, in generating system performance
estimates a standard $C_{n}^{2}(z)$ profile may be assumed.  Likewise,
Equations \ref{tphase_n7} and \ref{tprphase_n7} respectively represent
the tilt covariance and piston-removed / tilt phase cross covariance
for two sources at arbitrary ranges and and locations within the
field.  These quantities may be computed using the same inputs.

Calculation of these covariance functions requires two numerical
integrations over the unit circle (i.e over $\boldsymbol{\rho}_{1}$
and $\boldsymbol{\rho}_{2}$).  Assuming this integration is performed
using $\mathcal{N}$ samples across the pupil, then computation of
$\left\langle \vphantom{\hat{\phi}_{a}}
\phi_{a}\left(R\boldsymbol{\rho}_{1}\right)
\phi_{b}\left(R\boldsymbol{\rho}_{2}\right)\right\rangle$ is an
$\mathcal{O}(\mathcal{N}^{4})$ calculation.  Recall from Equation
\ref{lgsotf} that calculation of the four dimensional structure
function $D_{\rm apl}\left(\boldsymbol{r}_{1},
\boldsymbol{r}_{2}\right)$ must be performed for
$\mathcal{O}(\mathcal{N}^{4})$ values of $\boldsymbol{r}_{1}$ and
$\boldsymbol{r}_{2}$, each of which requires computation of 36
covariance functions.  This indicates that computation of the optical
transfer function is an $\mathcal{O}(36\mathcal{N}^{8})$ process.
Such a calculation is not a realistic proposition for general use in
astronomical data reduction.

\section{Computational Formulation}
\label{sec:compcovariance}

Tyler addresses the issue of computability by providing simplifying
expressions for each of the integrals that appear in Equations
\ref{prphase_n7}, \ref{tphase_n7} and \ref{tprphase_n7} .  This report
adheres to Tyler's Geugenbauer polynomial formalism, which is
presented in Appendix A3\cite{1994JOSAA..11..409T}.  The piston
removed phase covariance is given by
\begin{equation}\label{phasecovariance}
\left\langle \phi_{a}\left(R\boldsymbol{\rho}_{1}\right)  \phi_{b}\left(R\boldsymbol{\rho}_{2}\right)\right\rangle = 
\Xi D^{5/3} k^{2} \left\{
\mathcal{A}_{ab}\left(\boldsymbol{\rho}_{1}\right) + 
\mathcal{B}_{ab}\left(\boldsymbol{\rho}_{2}\right) - 
\mathcal{C}_{ab}\left(\boldsymbol{\rho}_{1},\boldsymbol{\rho}_{2}\right) - 
\mathcal{D}_{ab}\right\}
\end{equation}
and the tilt covariance is given by
\begin{equation}\label{tiltcovariance}
\begin{split}
\left\langle \phi^{T}_{a}\left(R\boldsymbol{\rho}_{1}\right)  \phi^{T}_{b}\left(R\boldsymbol{\rho}_{2}\right)\right\rangle & = 
\Xi D^{5/3} k^{2} \left\{
\left[\boldsymbol{\rho}_{1} \cdot \boldsymbol{\rho}_{2} \right] \mathcal{E}_{ab} + 
\mathcal{F}_{ab}\left(\boldsymbol{\rho}_{1}, \boldsymbol{\rho}_{2}\right)
\right\} 
\end{split}
\end{equation}
Finally, the cross covariance is given by
\begin{equation}\label{crosscovariance}
\begin{split}
\left\langle \phi_{a}\left(R\boldsymbol{\rho}_{1}\right)  \phi^{T}_{b}\left(R\boldsymbol{\rho}_{2}\right)\right\rangle & = 
\Xi D^{5/3} k^{2} \left\{
\left[\boldsymbol{\rho}_{1} \cdot \boldsymbol{\rho}_{2} \right] \mathcal{G}_{ab}\left(\boldsymbol{\rho}_{1}\right) + 
\mathcal{H}_{ab}\left(\boldsymbol{\rho}_{1}, \boldsymbol{\rho}_{2}\right) + 
\mathcal{I}_{ab}\left(\boldsymbol{\rho}_{2}\right)
\right\} 
\end{split}
\end{equation}
In these expressions, the following definitions have been employed.
\begin{equation}\label{caliph_A}
\mathcal{A}_{ab}\left(\boldsymbol{\rho}_{1}\right) = 
\int_{0}^{\mathcal{R}_{\rm min}} dz \; C_{n}^{2}\left(z\right) \left( 1 - \frac{z}{\mathcal{R}_{a}}\right)^{5/3}\; 
\left[Q_{ab}\left(z\right)\right]^{5/3} 
F_{1}\left(\frac{\left\vert\boldsymbol{\rho}_{1} + \boldsymbol{\Omega}_{ab}\left(z\right)\right\vert}{Q_{ab}\left(z\right)}\right)
\end{equation}
\begin{equation}\label{caliph_B}
\mathcal{B}_{ab}\left(\boldsymbol{\rho}_{2}\right) = 
\int_{0}^{\mathcal{R}_{\rm min}} dz \; C_{n}^{2}\left(z\right) \left( 1 - \frac{z}{\mathcal{R}_{a}}\right)^{5/3}\; 
F_{1}\left(\left\vert Q_{ab}\left(z\right)\boldsymbol{\rho}_{2} - \boldsymbol{\Omega}_{ab}\left(z\right)\right\vert\right)
\end{equation}
\begin{equation}\label{caliph_C}
\mathcal{C}_{ab}\left(\boldsymbol{\rho}_{1}, \boldsymbol{\rho}_{2}\right) = 
\int_{0}^{\mathcal{R}_{\rm min}} dz \; C_{n}^{2}\left(z\right) \left( 1 - \frac{z}{\mathcal{R}_{a}}\right)^{5/3}\; 
\left\vert\boldsymbol{\rho}_{1} - Q_{ab}\left(z\right)\boldsymbol{\rho}_{2} + \boldsymbol{\Omega}_{ab}\left(z\right)\right\vert^{5/3}
\end{equation}
\begin{equation}\label{caliph_D}
\mathcal{D}_{ab} = 
\int_{0}^{\mathcal{R}_{\rm min}} dz \; C_{n}^{2}\left(z\right) \left( 1 - \frac{z}{\mathcal{R}_{a}}\right)^{5/3}\; 
F_{2}\left(Q_{ab}\left(z\right),\left\vert\boldsymbol{\Omega}_{ab}\left(z\right)\right\vert\right)
\end{equation}
\begin{equation}\label{caliph_E}
\begin{split}
\mathcal{E}_{ab} = 
16 \int_{0}^{\mathcal{R}_{\rm min}} dz & \; C_{n}^{2}\left(z\right) \left( 1 - \frac{z}{\mathcal{R}_{a}}\right)^{5/3}\; 
\left[Q_{ab}\left(z\right)\right]^{-3}
\\ 
& \left\{\int_{0}^{{\rm max}\left(Q_{ab}\left(z\right)-\left\vert \boldsymbol{\Omega}_{ab} \left(z\right)\right\vert,0\right)} d\rho \; \rho^{3} \; \hat{G}\left(\rho\right) + 
\right.
\\ 
& 
\quad\quad
\frac{1}{\pi} \int^{Q_{ab}\left(z\right)+\left\vert \boldsymbol{\Omega}_{ab}\left(z\right)\right\vert}_{\left\vert Q_{ab}\left(z\right)-\left\vert \boldsymbol{\Omega}_{ab}\left(z\right)\right\vert\right\vert}
d\rho \; \rho^{3} \; \hat{G}\left(\rho\right) L\left(\left\vert\boldsymbol{\Omega}_{ab}\left(z\right)\right\vert, \rho, Q_{ab}\left(z\right)\right) \\
& 
\quad\quad\quad\quad
\left.
\sqrt{1 - \left[L\left(\left\vert\boldsymbol{\Omega}_{ab}\left(z\right)\right\vert, \rho, Q_{ab}\left(z\right)\right)\right]^{2}}
\vphantom{\int_{0}^{{\rm max}\left(Q_{ab}\left(z\right)-\left\vert \boldsymbol{\Omega}_{ab} \left(z\right)\right\vert,0\right)}}
\right\} 
\end{split}
\end{equation}
\begin{equation}\label{caliph_F}
\begin{split}
\mathcal{F}_{ab}\left(\boldsymbol{\rho}_{1}, \boldsymbol{\rho}_{2}\right) = 
\frac{32}{\pi}& \int_{0}^{\mathcal{R}_{\rm min}} dz \; C_{n}^{2}\left(z\right) \left( 1 - \frac{z}{\mathcal{R}_{a}}\right)^{5/3}
\left[\boldsymbol{\rho}_{1} \cdot \boldsymbol{\Omega}_{ab}\left(z\right) \right]
\left[\boldsymbol{\rho}_{2} \cdot \boldsymbol{\Omega}_{ab}\left(z\right) \right] \\
&
\int^{Q_{ab}\left(z\right)+\left\vert \boldsymbol{\Omega}_{ab}\left(z\right)\right\vert}_{\left\vert Q_{ab}\left(z\right)-\left\vert \boldsymbol{\Omega}_{ab}\left(z\right)\right\vert\right\vert}
d\rho \; \rho^{2} \; \hat{G}\left(\rho\right) 
\left[L\left(\left\vert\boldsymbol{\Omega}_{ab}\left(z\right)\right\vert, \rho, Q_{ab}\left(z\right)\right) - \frac{\left\vert \boldsymbol{\Omega}_{ab}\left(z\right) \right\vert}{\rho}\right]\\
& 
\quad\quad\quad\quad\quad\quad
\left[1 - \left[L\left(\left\vert\boldsymbol{\Omega}_{ab}\left(z\right)\right\vert, \rho, Q_{ab}\left(z\right)\right)\right]^{2}\right]^{1/2}
\end{split}
\end{equation}
\begin{equation}\label{caliph_G}
\mathcal{G}_{ab}\left(\boldsymbol{\rho}_{1}\right) = 
4 \int_{0}^{\mathcal{R}_{\rm min}} dz \; C_{n}^{2}\left(z\right) \left( 1 - \frac{z}{\mathcal{R}_{a}}\right)^{5/3}\; 
\left[Q_{ab}\left(z\right)\right]^{2/3} 
\hat{G}\left(\left\vert \frac{\boldsymbol{\rho}_{1} + \boldsymbol{\Omega}_{ab}\left(z\right)}{Q_{ab}\left(z\right)}\right\vert\right)
\end{equation}
\begin{equation}\label{caliph_H}
\mathcal{H}_{ab}\left(\boldsymbol{\rho}_{1}, \boldsymbol{\rho}_{2}\right) = 
4 \int_{0}^{\mathcal{R}_{\rm min}} dz \; C_{n}^{2}\left(z\right) \left( 1 - \frac{z}{\mathcal{R}_{a}}\right)^{5/3}\; 
\left[Q_{ab}\left(z\right)\right]^{2/3} \left[\boldsymbol{\rho}_{1}\cdot\boldsymbol{\Omega}_{ab}\left(z\right)\right]
\hat{G}\left(\left\vert \frac{\boldsymbol{\rho}_{2} + \boldsymbol{\Omega}_{ab}\left(z\right)}{Q_{ab}\left(z\right)}\right\vert\right)
\end{equation}
\begin{equation}\label{caliph_I}
\begin{split}
\mathcal{I}_{ab}\left(\boldsymbol{\rho}_{2}\right) = & 
4 \int_{0}^{\mathcal{R}_{\rm min}} dz \; C_{n}^{2}\left(z\right) \left( 1 - \frac{z}{\mathcal{R}_{a}}\right)^{5/3}\; 
\frac{\left[\boldsymbol{\rho}_{2}\cdot\boldsymbol{\Omega}_{ab}\left(z\right)\right] }{\pi \left[Q_{ab}\left(z\right)\right]^{1/3} 
\left\vert\boldsymbol{\Omega}_{ab}\left(z\right)\right\vert} \\
& 
\quad\quad\quad\quad\quad\quad
\int^{Q_{ab}\left(z\right) +\left\vert \boldsymbol{\Omega}_{ab}\left(z\right)\right\vert}_{\left\vert Q_{ab}\left(z\right) - \left\vert\boldsymbol{\Omega}_{ab}\left(z\right)\right\vert\right\vert}
d\rho \rho^{2} \left\{1 - \left[L\left(\left\vert\boldsymbol{\Omega}_{ab}\left(z\right)\right\vert, \rho, Q_{ab}\left(z\right)\right)\right]^{2}\right\}^{1/2} 
\hat{G}\left(\rho\right)
\end{split}
\end{equation}

The following definitions have been employed in Equations \ref{caliph_A} - \ref{caliph_H}.
\begin{equation}\label{F_1_eqn}
F_{1}\left(x\right) = 
\left\{
\begin{aligned}
\frac{6}{11}\; \vphantom{F}_{2}F_{1}\left(-\frac{11}{6}, -\frac{5}{6};1; x^{2}\right) & \quad x\le 1 \\
x^{5/3}\; \vphantom{F}_{2}F_{1}\left(-\frac{5}{6}, -\frac{5}{6};2; x^{-2}\right)        & \quad x>1
\end{aligned}
\right.
\end{equation}
\begin{equation}\label{F_2_eqn}
F_{2}\left(\alpha, x\right) = \frac{2}{\pi \alpha^{2}} \int_{0}^{1+\alpha} d\rho K_{1}\left(\rho, \alpha\right)H\left(\rho, x\right)
\end{equation}
\begin{equation}\label{K_1_eqn}
K_{1}\left(\rho,\alpha\right) = 
\left\{
\begin{aligned}
&
\pi \alpha^{2} & \left\vert\rho\right\vert \le 1-\alpha \\
&
\begin{split}
\cos^{-1}&\left(t_{0}\right) - t_{0}\sqrt{1-t_{0}} + \\
&\alpha^{2}\left[\cos^{-1}\left(s_{0}/\alpha\right) - s_{0}/\alpha\sqrt{1-s_{0}/\alpha^{2}}\right] 
\end{split}
& \quad 1 - \alpha \le \left\vert\rho\right\vert  \le 1+\alpha \\
\\
& 0 & {\rm Otherwise}
\end{aligned}
\right.
\end{equation}
\begin{equation}\label{s0_t0_eqn}
\begin{aligned}
s_{0} = \frac{\rho}{2} - \frac{1 - \alpha^{2}}{2\rho} \quad & \quad t_{0} = \frac{\rho}{2} + \frac{1 - \alpha^{2}}{2\rho} 
\end{aligned}
\end{equation}
\begin{equation}\label{H_eqn}
H\left(\rho,x\right) = 
\left\{
\begin{aligned}
\rho^{8/3} \vphantom{F}_{2}F_{1}\left(-\frac{5}{6}, -\frac{5}{6};1; \left(x/\rho\right)^{-2}\right)        & \quad x\le \rho \\
\rho x^{5/3} \vphantom{F}_{2}F_{1}\left(-\frac{5}{6}, -\frac{5}{6};1; \left(\rho/x\right)^{-2}\right)        & \quad x > \rho
\end{aligned}
\right.
\end{equation}
\begin{equation}\label{G_hat_eqn}
\hat{G}\left(\alpha\right) = 
\left\{
\begin{aligned}
-\frac{5}{11} \vphantom{F}_{2}F_{1}\left(-\frac{11}{6}, \frac{1}{6};2; \alpha^{2}\right)              & \quad \alpha\le 1 \\
-\frac{5}{12} \alpha^{-1/3} \vphantom{F}_{2}F_{1}\left(-\frac{5}{6}, \frac{1}{6};3; \alpha^{-2}\right)  & \quad \alpha > 1
\end{aligned}
\right.
\end{equation}
\begin{equation}
L\left(a, b, c\right) = \frac{a^{2} + b^{2} - c^{2}}{2 a b}
\end{equation}

The above expressions appear complex, but provide significant
computational advantage over Equation \ref{prphase_n7}.  The
calculation of Equations \ref{caliph_A} - \ref{caliph_I} require at
most the calculation of a one dimensional integral over
$C_{n}^{2}(z)$, a one dimensional integral over another variable, and
computation of a hypergeometric function.  The latter computation is
easily terminated when contributions of successive hypergeometric
terms fall below a fixed precision.  This is vastly more
computationally efficient than the $\mathcal{O}(\mathcal{N}^{4})$
calculation required to compute Equations \ref{prphase_n7} -
\ref{tprphase_n7} directly.

The form of Equations \ref{phasecovariance} - \ref{crosscovariance}
permit further computational advantages.  Six of the nine expressions
display a reduced dependence on $\boldsymbol{\rho}_{1}$ and/or
$\boldsymbol{\rho}_{2}$, and may be precomputed and stored for use in
computing the covariance functions over the pupil plane.

In laser guide star adaptive optics systems, one is interested in
computing the field dependent target optical transfer function
${\rm OTF}_{\rm sci}\left(\boldsymbol{r}\right)$ assuming fixed locations of
the tip tilt and laser guide stars and for a specific turbulence
profile.  For example, if one was interested in modelling a crowded
field imaged by a science camera, one would wish to compute ${\rm OTF}_{\rm
  sci}\left(\boldsymbol{r}\right)$ at many locations within the field
of view.  In such an application, one would precompute the piston
removed phase and tilt covariance functions involving the laser and
tip tilt star that appear in Equation \ref{lgsstrfnexpansion}.  One
would then iterate over locations in the field, computing only the
terms in in Equation \ref{lgsstrfnexpansion} that depend on target
location.


\section{Anisoplanatic Structure Function}
\label{sec:strfn}

%\phi_{\rm sci}\left(\boldsymbol{r}\right)
%\phi^{T}_{\rm sci}\left(\boldsymbol{r}\right) 
%\phi^{T}_{\rm ttgs}\left(\boldsymbol{r}\right)
%\phi_{\rm lgs}\left(\boldsymbol{r}\right)
%\phi^{T}_{\rm lgs}\left(\boldsymbol{r}\right)

For computational purposes it is useful to explicitly expand the
structure function in Equation \ref{lgsstrfnexpansion} and group like
terms, yielding
\begin{equation}\label{strfn_x}
\small
\begin{aligned}
D_{\phi_{\rm apl}}&(\boldsymbol{r}_{1}, \boldsymbol{r}_{2}) = 
\\ &
\left\langle\left\{\phi_{\rm sci}\left(\boldsymbol{r}_{1}\right)\right\}^{2}\right\rangle +
\left\langle\left\{\phi_{\rm sci}\left(\boldsymbol{r}_{2}\right)\right\}^{2}\right\rangle -
2\left\langle\phi_{\rm sci}\left(\boldsymbol{r}_{1}\right) \phi_{\rm sci}\left(\boldsymbol{r}_{2}\right)\right\rangle + 
\\ & 
\left\langle\left\{\phi^{T}_{\rm ttgs}\left(\boldsymbol{r}_{1}\right)\right\}^{2}\right\rangle  + 
\left\langle\left\{\phi^{T}_{\rm ttgs}\left(\boldsymbol{r}_{2}\right) \right\}^{2}\right\rangle -
2\left\langle\phi^{T}_{\rm ttgs}\left(\boldsymbol{r}_{1}\right) \phi^{T}_{\rm ttgs}\left(\boldsymbol{r}_{2}\right)\right\rangle +
\\ & 
\left\langle\left\{ \phi_{\rm lgs}\left(\boldsymbol{r}_{1}\right)\right\}^{2}\right\rangle +
\left\langle\left\{ \phi_{\rm lgs}\left(\boldsymbol{r}_{2}\right)\right\}^{2}\right\rangle -
2\left\langle\phi_{\rm lgs}\left(\boldsymbol{r}_{1}\right) \phi_{\rm lgs}\left(\boldsymbol{r}_{2}\right)\right\rangle +
\\ & 
\left\langle\left\{ \phi^{T}_{\rm lgs}\left(\boldsymbol{r}_{1}\right)\right\}^{2}\right\rangle  +
\left\langle\left\{ \phi^{T}_{\rm lgs}\left(\boldsymbol{r}_{2}\right)\right\}^{2}\right\rangle - 
2\left\langle\phi^{T}_{\rm lgs}\left(\boldsymbol{r}_{1}\right) \phi^{T}_{\rm lgs}\left(\boldsymbol{r}_{2}\right)\right\rangle +
\\ & 
2\left[-\left\langle\phi_{\rm sci}\left(\boldsymbol{r}_{1}\right) \phi^{T}_{\rm ttgs}\left(\boldsymbol{r}_{1}\right)\right\rangle + 
\left\langle\phi_{\rm sci}\left(\boldsymbol{r}_{2}\right) \phi^{T}_{\rm ttgs}\left(\boldsymbol{r}_{2}\right)\right\rangle +
\left\langle\phi_{\rm sci}\left(\boldsymbol{r}_{1}\right) \phi^{T}_{\rm ttgs}\left(\boldsymbol{r}_{2}\right)\right\rangle +
\left\langle\phi_{\rm sci}\left(\boldsymbol{r}_{2}\right) \phi^{T}_{\rm ttgs}\left(\boldsymbol{r}_{1}\right)\right\rangle\right] +
\\ & 
2\left[-\left\langle\phi_{\rm sci}\left(\boldsymbol{r}_{1}\right) \phi_{\rm lgs}\left(\boldsymbol{r}_{1}\right)\right\rangle + 
\left\langle\phi_{\rm sci}\left(\boldsymbol{r}_{2}\right) \phi_{\rm lgs}\left(\boldsymbol{r}_{2}\right)\right\rangle +
\left\langle\phi_{\rm sci}\left(\boldsymbol{r}_{1}\right) \phi_{\rm lgs}\left(\boldsymbol{r}_{2}\right)\right\rangle +
\left\langle\phi_{\rm sci}\left(\boldsymbol{r}_{2}\right) \phi_{\rm lgs}\left(\boldsymbol{r}_{1}\right)\right\rangle\right] +
\\ & 
2\left[\left\langle\phi_{\rm sci}\left(\boldsymbol{r}_{1}\right) \phi^{T}_{\rm lgs}\left(\boldsymbol{r}_{1}\right)\right\rangle -
\left\langle\phi_{\rm sci}\left(\boldsymbol{r}_{2}\right) \phi^{T}_{\rm lgs}\left(\boldsymbol{r}_{2}\right)\right\rangle -
\left\langle\phi_{\rm sci}\left(\boldsymbol{r}_{1}\right) \phi^{T}_{\rm lgs}\left(\boldsymbol{r}_{2}\right)\right\rangle -
\left\langle\phi_{\rm sci}\left(\boldsymbol{r}_{2}\right) \phi^{T}_{\rm lgs}\left(\boldsymbol{r}_{1}\right)\right\rangle\right] + 
\\ & 
2\left[\left\langle\phi^{T}_{\rm ttgs}\left(\boldsymbol{r}_{1}\right) \phi_{\rm lgs}\left(\boldsymbol{r}_{1}\right)\right\rangle - 
\left\langle\phi^{T}_{\rm ttgs}\left(\boldsymbol{r}_{2}\right) \phi_{\rm lgs}\left(\boldsymbol{r}_{2}\right)\right\rangle -
\left\langle\phi^{T}_{\rm ttgs}\left(\boldsymbol{r}_{1}\right) \phi_{\rm lgs}\left(\boldsymbol{r}_{2}\right)\right\rangle -
\left\langle\phi^{T}_{\rm ttgs}\left(\boldsymbol{r}_{2}\right) \phi_{\rm lgs}\left(\boldsymbol{r}_{1}\right)\right\rangle\right] + 
\\ & 
2\left[-\left\langle\phi^{T}_{\rm ttgs}\left(\boldsymbol{r}_{1}\right) \phi^{T}_{\rm lgs}\left(\boldsymbol{r}_{1}\right)\right\rangle + 
\left\langle\phi^{T}_{\rm ttgs}\left(\boldsymbol{r}_{2}\right) \phi^{T}_{\rm lgs}\left(\boldsymbol{r}_{2}\right)\right\rangle +
\left\langle\phi^{T}_{\rm ttgs}\left(\boldsymbol{r}_{1}\right) \phi^{T}_{\rm lgs}\left(\boldsymbol{r}_{2}\right)\right\rangle +
\left\langle\phi^{T}_{\rm ttgs}\left(\boldsymbol{r}_{2}\right) \phi^{T}_{\rm lgs}\left(\boldsymbol{r}_{1}\right)\right\rangle\right] +
\\ & 
2\left[-\left\langle\phi_{\rm lgs}\left(\boldsymbol{r}_{1}\right) \phi^{T}_{\rm lgs}\left(\boldsymbol{r}_{1}\right)\right\rangle + 
\left\langle\phi_{\rm lgs}\left(\boldsymbol{r}_{2}\right) \phi^{T}_{\rm lgs}\left(\boldsymbol{r}_{2}\right)\right\rangle +
\left\langle\phi_{\rm lgs}\left(\boldsymbol{r}_{1}\right) \phi^{T}_{\rm lgs}\left(\boldsymbol{r}_{2}\right)\right\rangle +
\left\langle\phi_{\rm lgs}\left(\boldsymbol{r}_{2}\right) \phi^{T}_{\rm lgs}\left(\boldsymbol{r}_{1}\right)\right\rangle\right]
\end{aligned}
\end{equation}
This equation naturally lends itself to a C++ programming interface in
which a phase covariance function such as $\left\langle\phi^{T}_{\rm
  ttgs}\left(\boldsymbol{r}_{1}\right) \phi^{T}_{\rm
  ttgs}\left(\boldsymbol{r}_{2}\right)\right\rangle$ is represented as
an instance of a class {\bf phase\_covariance}, with a member function
\\ 
\indent {\bf double phase\_covariance::covariance(vec r\_1, vec r\_2)}
\\
The class instance may store partial calculations specific to the
source location, aperture diameter, and $C_{n}^{2}$ profile.  For
example, the value of $\mathcal{D}_{ab}$ may be computed at
instantiation, while the values of $\mathcal{A}_{ab}\left(\rho\right)$
and $\mathcal{B}_{ab}\left(\rho\right)$ may be computed as a function
of $\rho$ upon first invocation of the above member function.
Likewise, classes {\bf tilt\_covariance} and {\bf phase\_tilt\_covariance}
may represent the other types of covariance functions in Equation
\ref{strfn_x}.  

%\section{Example: Differential Phase Variance}
%\label{sec:variance}

%\begin{equation}\label{uncomp_variance_eqn}
%\end{equation}

%\begin{equation}\label{uncomp_tr_variance_eqn}
%\end{equation}

%\begin{equation}\label{ngs_variance_eqn}
%\end{equation}

%\begin{equation}\label{ngs_tr_variance_eqn}
%\end{equation}

%\begin{equation}\label{lgs_variance_eqn}
%\end{equation}

%\begin{equation}\label{lgs_tr_variance_eqn}
%\end{equation}


%\section{Example: Structure Functions}
%\label{sec:strfn}



%\section{Example: Point Spread Functions}
%\label{sec:psf}




%\section{IDL Bindings}
%\label{sec:idl}
