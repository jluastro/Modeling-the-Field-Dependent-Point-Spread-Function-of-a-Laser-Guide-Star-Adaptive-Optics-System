Here is my figure caption copied from Matthew's note. Figures 1a and 1b show a possible geometrical configuration for the tip tilt, target, and laser. Light from these targets traverses atmospheric turbulence along different paths, thereby acquir- ing differential wavefront aberrations. This is the effect of anisoplanatism, which arises from the finite vertical extent of atmospheric turbulence. Angular anisoplanatism represents the difference between the wavefront aberrations from sources at different locations in the field of view, while focal anisoplanatism represents the difference in wavefront aberrations from sources at different ranges from the telescope aperture. Both types of anisoplanatism are present in this geometrical configuration.
\label{fig:guide_star_schematic}
