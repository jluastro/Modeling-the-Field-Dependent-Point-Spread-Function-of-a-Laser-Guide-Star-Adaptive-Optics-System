\maketitle
\section{Introduction}

Adaptive optics (AO) systems on large (8-10 m)
ground-based telescopes, such as at the W.~M.~Keck Observatory,
provide some of the sharpest infrared astronomical images. 
High-resolution AO images and spectroscopy have produced 
notable scientific results, including precise measurements of the mass
and distance of the supermassive black hole at the Galactic Center. 
AO images of stars orbiting close to this supermassive black
hole have the potential to uniquely test general relativity in the strong
gravity regime. However, the potential of current AO systems has not been
fully exploited in terms of astrometric and photometric precision due
to our incomplete knowledge of the temporal and spatial variations in
the point spread function (PSF). 

The current Keck II AO system is equipped with a laser guide star 


Motivated by the need for improved
astrometry and photometry on Galactic Center observations taken by the
Keck telescopes' AO systems, we have developed new algorithms bundeled
in the AIROPA package ({\it A}nisoplanatic and {\it I}nstrumental {\it
  R}econstruction of {\it O}ff-axis {\it P}SFs for {\it A}O) to
account for the spatial and temporal variations of the PSF. 
In this paper, we describe the 


Fubar... this is the introduction. \cite{Britton:2006}


