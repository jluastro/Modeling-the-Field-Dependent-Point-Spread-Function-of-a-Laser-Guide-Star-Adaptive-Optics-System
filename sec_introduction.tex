\section{Introduction}

Adaptive optics (AO) systems on large (8-10 m)
ground-based telescopes, such as at the W.~M.~Keck Observatory (WKMO),
provide some of the sharpest infrared astronomical images. 
High-resolution AO images and spectroscopy from WMKO have produced 
notable scientific results, including precise measurements of the mass
and distance of the supermassive black hole at the Galactic Center
\cite{Ghez:2008,Meyer:2012}, 
directly imaging exoplanets \cite{Marois:2008,Marois:2010},
constraining the progenitors of exotic supernovae
\cite{Smith:2007,GalYam:2007},
and mapping weather and volcanos on moons in the solar system
\cite{Roe:2002,Marchis:2005}
Despite their success, current AO systems have not yet reached their
full potential due to our incomplete knowledge of the temporal and
spatial variations in the point spread function (PSF,
\cite{Davies:2012,Lu:2014}). 
Current astrometric and photometric precisions have not reached 
the signal-to-noise limit and instead reach a systematic error floor
due to errors in PSF estimation.

Modeling the adaptive optics PSF and its spatial variation has
been developed in theory 
\cite{Veran:1996,Flicker:2003,Steinbring:2003,Steinbring:2005,Gendron:2006} 
and has been attempted both in simulations and on-sky with some success 
\cite{Harder:1998,Harder:1999,Jolissaint:2004,Aubailly:2005,Britton:2006,Clenet:2008,Martin:2016}.
Nearly all of the on-sky studies were conducted using natural guide
stars and resulted in $>$20\% errors. To date, no PSF reconstruction tool is
used regularly on astronomical adaptive optics observations.
\begin{notes}
CHECK THIS
\end{notes}
The primary limitation has been the lack of a robust end-user tool
that can easily ingest the wide range of calibration, atmospheric
turbulence, AO telemetry, and scientific data and produce a precise grid
of model PSFs that are then used to extract photometry, astrometry,
and morphology information on the science target. And all of this
needs to be performed in a computationally efficient manner.



% \cite{Steinbring:2005,Davies:2012,Trippe:2010}. 

% \cite{Mao:2016}
% Refinements in PSF prediction.

% \cite{Flicker:2003}
% New model for reconstructing tilt anisoplanatism.

% \cite{Veran:1996}
% First proposal of using telemetry to reconstruct the on-axis PSF.

% \cite{Harder:1998,Harder:1999}
% First application to SH WFS on ADONIS (on faint targets).

% \cite{Jolissaint:2004}
% Applied Veran method to Altair. (NGS)

% \cite{Aubailly:2005}
% anisoplanatic PSF reconstruction... compared to simulations only.

% \cite{Gendron:2006}
% New computational algorithms for PSF reconstruction.

% \cite{Clenet:2008}
% Testing on NACO

% \cite{Martin:2016}
% Testing on canary

The need for improved PSF modeling in adaptive optics images is
wide-spread; however, it is particularly acute in the case of Galactic
Center studies. Adaptive optics measurements of stars orbiting the
supermassive black hole can be used to test General
Relativity and other theories of gravity in a strong-field, high-mass
regime \cite{Hees:2017}. To achieve this scientific goal, improvements in the
astrometric and photometric precision are required. Furthermore, the
astrometry must be improved over the entire field of view since all
many stars are needed to establish an accurate absolute cooridinate
system that doesn't move and distort with time or changes in the
atmospheric conditions.

Motivated by the need for improved astrometry and photometry in
Keck adaptive optics observations of the Galactic Center, we have developed new
algorithms to reconstruct the spatial variation of the point-spread
function (PSF). 
These algorithms are deployed in a software package called
 AIROPA ({\it A}nisoplanatic and {\it I}nstrumental {\it
  R}econstruction of {\it O}ff-axis {\it P}SFs for {\it A}O) to
account for both the spatial and temporal variations of the PSF
\cite{Fitzgerald:2012,Sitarski:2014,Witzel:2016}. 
AIROPA accounts for PSF variations due to both atmospheric
anisoplanatism \cite{Fitzgerald:2012} and instrumental abberrations
\cite{Sitarski:2014}. The modeling framework for the
atmospheric anisoplanatism is based on a successful experiment
conducted at Palomar that uses external measurements of the $C_n^2$
profile of the atmospheric turbulence from a MASS/DIMM unit
\cite{Britton:2006}.  This experiment and the underlying model
algorithms were develop for natural guide star. In this paper, 
we present the mathematical formalism and implementation details for
modeling laser guide star observations with AIROPA. 

\section{AIROPA for NIRC2 on Keck II}

The AIROPA project is initially focused on modeling PSF variation in
observations obtained with the Keck II AO system \cite{Wizinowich:2006,vanDam:2006} 
and the NIRC2 camera (PI: K. Matthews). To model the PSF variation, a
number of telescope and adaptive optics system parameters are
required, which are listed in Table \ref{tab:keck_ao_nirc2}.
These parameters are stored in an instrumental configuration file,
which can be passed into AIROPA. As part of this configuration,
a map of the spatially dependent instrumental aberrations are needed
as described in \cite{Sitarski:2014}. Incorporation of the instrumental 
aberrations is described in more detail below.

\begin{deluxetable}{llp{3in}}
\tabletypesize{\footnotesize}
\tablewidth{0pt}
\tablecaption{Keck AO and NIRC2 Parameters}
\tablehead{
Parameter & Value & Description 
}
\startdata
Instrument: & NIRC 2 & The name of the instrument. Currently only NIRC2 is supported. \\
Pixel scale as: & 0.01 & The instrument pixel scale in units of mas/pixel. \\
Tel pupil size: & 11.14 & The diameter of the primary mirror in m. \\
Ph map pupil size pix: & 65 & The number of pixels across the instrumental phase maps. \\
Pupil: & NIRC 2 large hex pupil generator & The NIRC2 pupil mask used. Options include (). \\
Phase map model: & phase\textunderscore cube\textunderscore old\textunderscore 251.fits & The name of the FITS file 
that contains the grid of OTF ratios for the instrumental model. \\
X positions: & xvals.fits & The X positions for each OTF ratio array in the FITS files for the instrumental model. \\ 
Y positions: & yvals.fits & The Y positions for each OTF ratio array in the FITS files for the instrumental model. \\
AO mode: & LGS & Either LGS or NGS. \\
\enddata
% \tablenotetext{}{test}
% A description of the instrumental model grid of OTF ratio arrays is
% given in \cite{Sitarski:2014}.
% }
\label{tab:keck_ao_nirc2}
\end{deluxetable}



