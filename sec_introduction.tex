\section{Introduction}

Adaptive optics (AO) systems on large (8-10 m)
ground-based telescopes, such as at the W.~M.~Keck Observatory,
provide some of the sharpest infrared astronomical images. 
High-resolution AO images and spectroscopy have produced 
notable scientific results, including precise measurements of the mass
and distance of the supermassive black hole at the Galactic Center \cite{Ghez:2008}. 
AO images of stars orbiting close to this supermassive black
hole have the potential to uniquely test general relativity in the strong
gravity regime. However, the potential of current AO systems has not been
fully exploited in terms of astrometric and photometric precision due
to our incomplete knowledge of the temporal and spatial variations in
the point spread function \cite[PSF][]{Davies:2012,Lu:2014}. 
Current astrometric and photometric limits are 
\begin{notes}
Add more here.
\end{notes}

Motivated by the need for improved astrometry and photometry in
adaptive optics observations of crowded-fields, we have developed new
algorithms to reconstruct the spatial variation of the PSF. 
These algorithms are deployed in a software package called
 AIROPA ({\it A}nisoplanatic and {\it I}nstrumental {\it
  R}econstruction of {\it O}ff-axis {\it P}SFs for {\it A}O) to
account for the spatial and temporal variations of the PSF. 
\begin{notes}
CROSS REF to Gunther's papers
\end{notes}
In this paper, we describe the 

The current Keck II AO system is equipped with a laser guide star,
single-conjugate adaptive optics system 

More PSF reconstruction references \cite{Steinbring:2005,Davies:2012,Trippe:2010}

Fubar... this is the introduction. \cite{Britton:2006}


