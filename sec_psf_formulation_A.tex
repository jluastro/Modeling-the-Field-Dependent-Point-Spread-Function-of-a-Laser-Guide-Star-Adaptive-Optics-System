\section{Point Spread Function Formulation}
\label{sec:psfform}

Consider a single-conjugate, laser guide star adaptive optics
(LGS-AO) system, where a natural guide
star, located at an angular sky position, $\vec{\theta}_{ttgs}$,
 is used for
tilt sensing and compensation and the laser, located at an angular sky position,
$\vec{\theta}_{lgs}$, is used for high-order sensing and compensation. 
The science target of interest is offset from both the laser and the
tip-tilt guide star at position, $\vec{\theta}_{sci}$.  
Both the tip-tilt guide star (TTGS) used and the science
target are effectively at infinite range while the laser guide star is
at $\sim$90 km. 
Figure \ref{fig:guide_star_schematic} displays schematic configuration for
such an AO system and observation.
Light from each of these sources descends
through columns of atmosphere to arrive at the aperture of the
telescope.  During their passage through the atmosphere, wavefronts
acquire phase aberrations due to turbulence-induced fluctuations in
the index of refraction of air \cite{Kolmogorov:1941}.
These wavefronts arrive at the telescope, where they are sensed by
detectors. The TTGS wavefront is sensed with a low-order tracker to determine wavefront
tilt, while the wavefront from the laser is sensed by a high-order 
wavefront sensor.  




